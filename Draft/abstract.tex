\section{Abstract}
\label{sec:abstract}
Context: Researchers and practitioners have proposed and evaluated a wide variety metrics for measuring software and software development process characteristics and their impact on software security. However, assessing software security requires measuring not only software and development process characteristics, but also the context in which the software is used. 

Goal:  \textit{The goal of this research is to support researchers and practitioners in measuring the effect of software and environmental context factors and security practice adherence on software security outcomes by building and evaluating an explanatory model using structural equation modeling.} 

Method: In this work, we develop a model hypothesizing relationships between software context factors(Software Risk), software usage factors(Asset Impact), adherence(Adherence) to security practices, and security outcomes (Outcomes).  We evaluated the model and our hypotheses quantitatively, applying Structural Equation Modeling (SEM) to two existing datasets: 6695 projects sampled from the National Vulnerability Database (NVD), and the 428 projects from the Core Infrastructure Initiative (CII) census dataset.  

Results: As measured in our data from both case studies, Software Risk, Asset Impact, and Adherence each have statistically significant effects on Outcomes. The NVD case study results indicate that software security has declined between 2001 and 2016, as measured by CVSS access and impact metrics, although we did find evidence for increasing access complexity being correlated with lower vulnerability counts. The CII case study results agreed with previous research suggesting that team size and code size are among the most significant influences on vulnerabilities. We found that higher package popularity was correlated with lower vulnerability counts. In both studies, network exposure is correlated with higher vulnerability counts.

Conclusion: The model and results from the two case studies offer a guide to available metrics for prioritizing management of Software Risk, Asset Impact, Adherence measures, and security Outcomes. SEM offers means for assessing model fit and iteratively refining the model to more accurately model what influences security outcomes.