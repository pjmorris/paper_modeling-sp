%\section{Abstract}
\label{sec:abstract}
\textbf{Abstract} 
Context: Researchers and practitioners have proposed and evaluated metrics for measuring software characteristics and software development process characteristics and their impact on software security. However, assessing software security also requires measuring the context in which the software is used. 
Goal:  \textit{The goal of this research is to support researchers and practitioners in measuring the effect of software context factors and security practice adherence on software security outcomes by building an explanatory model, and evaluating the model using structural equation modeling.} 

Method: We develop a model hypothesizing relationships between software context factors, security practice adherence, and security outcomes.  We evaluated the model and our hypotheses quantitatively, applying Structural Equation Modeling (SEM) to 697 projects reported on in two public datasets,  OpenHub, and the National Vulnerability Database (NVD).

Results: In our data, vulnerability counts are almost as strongly correlated with software usage context factors (0.34, as measured by user count) as they are correlated with traditional software development context factors (0.40, as measured by code size, code churn, team size, and project age), illustrating the importance of software usage context when considering software security outcomes. Practice adherence, as measured by the ratio of developer count to code churn over the previous 12 months, was negatively correlated ($-0.57$)) with vulnerability counts. The case study results agreed with previous research suggesting that team size and code churn are correlated with worse security outcomes, as measured by vulnerability counts. 

Conclusion: The model and case study results offer a guide to available metrics for assessing the security impact of software context factors and software usage factors. SEM offers a means for assessing model fit and iteratively refining the model to more accurately model influences on security outcomes.