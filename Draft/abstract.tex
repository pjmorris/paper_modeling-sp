\section{Abstract}
\label{sec:abstract}
Researchers and practitioners have proposed and evaluated many practices for measuring and improving software security during software development. However, vulnerabilities may be found by attackers in released software, even in very carefully written software. Assessing the effectiveness of security practices in software development projects requires accounting for vulnerability risk; we might ask whether `secure' software is secure because of the care taken in its construction, or because it is lucky in its enemies. \textit{The goal of this research is to support researchers and practitioners in measuring the effect of security practice use on software vulnerabilities by building and validating an explanatory model for software development security practice use.} In this work, we specify a model hypothesizing relationships between security criticality, security effort, and security outcomes.  We evaluate the model using empirical data drawn from case studies of two open source projects, OpenSSL and phpMyAdmin. In our data, security outcomes were influenced by both the context of the software and the influence of security practice use. The strongest relationships between security practice use and favorable security outcomes were associated with team use of vulnerability tracking, security review, and secure coding standards. 