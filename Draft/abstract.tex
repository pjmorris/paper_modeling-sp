\section{Abstract}
\label{sec:abstract}
Context: Researchers and practitioners have proposed and evaluated many practices for measuring and improving software security during software development. However, we have little published empirical assessment of how development team adoption of these security practices influence security outcomes in software. Further, environmental factors influence security outcomes for software. Assessing security outcomes for software requires measuring both the effectiveness of security practices in software development and the vulnerability risk for the released software.

Goal:  \textit{The goal of this research is to support researchers and practitioners in measuring the effect of security practice use on software vulnerabilities by building and validating an explanatory model for software development security practice use.} 

Method: In this work, we specify a model hypothesizing relationships between security risk, adherence to 13 security practices, and security outcomes.  We evaluate the model quantitatively, via the Core Infrastructure Initiative census dataset of security-critical projects, and qualitatively, via case studies of two open source projects, OpenSSL and phpMyAdmin. 

Results: In our data, security outcomes were influenced by both software risk and security practice adherence. The strongest relationships between security practice use and favorable security outcomes were associated with team adherence to the practices of vulnerability tracking, security review, and secure coding standards. The weakest relationships were... Conclusion: Measuring security practice adherence supports software quality and software risk management.