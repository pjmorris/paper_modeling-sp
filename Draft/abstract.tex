\section{Abstract}
\label{sec:abstract}

Software development teams seek to minimize vulnerabilities in the software they produce. Many practices for measuring and improving software security have been proposed and evaluated.  However, quantifying the effectiveness of security practices in software development projects remains an active research area. A model for how software-development-security-practice use affects security outcomes would allow assessment of the degree to which data and theory agree, and would provide indicators of the relative importance of each element modeled. \textit{The goal of this research is to assist software engineering researchers by specifying and evaluating a quantitative model for how software development security practice use affects security outcomes.} In this work, we specify a theoretical model hypothesizing relationships between security criticality, security effort, and security outcomes.  We evaluate the model using Structural Equation Modeling and empirical data extracted from the Common Infrastructure Initiative. We collected measurements for 242 projects,  and used the data to estimate the parameters of our theoretical model, allowing us to test the strength of our hypotheses. In our data, the strongest relationships between security practice use and favorable security outcomes were associated with team use of vulnerability tracking, security review, and secure coding standards. 

