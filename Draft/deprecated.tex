
%
% The code below should be generated by the tool at
% http://dl.acm.org/ccs.cfm
% Please copy and paste the code instead of the example below. 
%
\begin{CCSXML}
	<ccs2012>
	<concept>
	<concept_id>10010520.10010553.10010562</concept_id>
	<concept_desc>Computer systems organization~Embedded systems</concept_desc>
	<concept_significance>500</concept_significance>
	</concept>
	<concept>
	<concept_id>10010520.10010575.10010755</concept_id>
	<concept_desc>Computer systems organization~Redundancy</concept_desc>
	<concept_significance>300</concept_significance>
	</concept>
	<concept>
	<concept_id>10010520.10010553.10010554</concept_id>
	<concept_desc>Computer systems organization~Robotics</concept_desc>
	<concept_significance>100</concept_significance>
	</concept>
	<concept>
	<concept_id>10003033.10003083.10003095</concept_id>
	<concept_desc>Networks~Network reliability</concept_desc>
	<concept_significance>100</concept_significance>
	</concept>
	</ccs2012>  
\end{CCSXML}



Statistician Andrew Gelman once critiqued a study as follows `Their effect size is tiny and their measurement error is huge. My best analogy is that they are trying to use a bathroom scale to weigh a feather - and the feather is resting loosely in the pouch of a kangaroo that is vigorously jumping up and down.' ~\footnote{http://andrewgelman.com/2015/04/21/feather-bathroom-scale-kangaroo/} We suspect that measuring the effects of security practice application in software projects is prone to similar difficulties, given the wide variety of software development processes and software applications. To measure the effect of security practices on vulnerabilities, we have to account for the other influences on vulnerabilities. To do so, we build a model of the variables we expect to influence security outcomes, and to account for the various factors that influence software project performance.


As evidenced by the Heartbleed~\footnote{\url{http://heartbleed.com/}} and Shellshock~\footnote{\url{http://tinyurl.com/mfv93td}} vulnerabilities, a single commit, sometimes even a single line of code, can cause a large-scale security problem.  Consequently, software development teams are investigating the use of security practices in software development.

 When presented with lists of the dozens of software development security practices available (e.g. those listed in the `Building Security in Maturity Model ~\cite{mcgraw2013bsimm} (BSIMM)), the first question industry practitioners typically ask is `What security practices should we  adopt?'  To answer this question requires an understanding of the questioner's security problems, and a theory for how software development security practices affect the software. To support collection of the data needed to answer questions about security practices and their effects empirically, we present a model for how security practice use affects security outcomes in software development.  Ultimately, our goal is to provide evidence-based answers for questions like `What security practices should we adopt?', and `What do we do next?'.
 
 A number of organizations have published lists of security practices, including the Building Security in Maturity Model~\cite{mcgraw2013bsimm} (BSIMM), the Microsoft Security Development Lifecycle~\cite{howard2009security} (SDL), the Software Assurance Forum for Excellence in Code~\cite{simpson2013fundamental} (SAFECode), and the Open Web Application Security Project (OWASP) Software Security Assurance Process ~\cite{martinez2014ssap} (SSAP). Organizations and development teams may use one or more of these lists, however the lists do not specify how to go about choosing, implementing, monitoring, and assessing the effects of a set of security practices.
 Mockus and Votta~\cite{mockus2000identifying} used keyword matching to distinguish between new code development, corrective, and perfective maintenance activities in version control system commit logs.

Typical definitions of risk include measures of the size of the potential threat, and the probability that the threat will occur. For example, the NIST recently defined risk ~\cite{nist2011managing} as: `A measure of the extent to which an entity is threatened by a potential circumstance or event, and typically a function of: (i) the adverse impacts that would arise if the circumstance or event occurs; and (ii) the likelihood of occurrence.' In keeping with typical definitions of risk, we define two sub-constructs, Impact, and Likelihood. Impact represents the value an attacker will find in conducting a successful attack on the assets made available through the software.

Adherence represents the efforts the team takes to prevent and discover vulnerabilities. We adapt an IEEE definition of practice~\cite{ieee1990glossary} `3. a specific type of professional or management activity that contributes to 
the execution of a process and that may employ one or more techniques and tools' to define a software development security practice to be an action a software development team member takes to prevent, identify, or resolve a vulnerability, possibly guided by a tool or reference. From our definition, we define a template for describing security practice use: \textless Role\textgreater (team member)\textless Verb\textgreater (action)\textless Artifact(s) Affected\textgreater, (guided by)\textless Artifact(s) Referenced\textgreater. Our definition and template allows description of the actions team members take (e.g. `Developer ran Findbugs before checking in code', `Manager documented threat model for new design') in the course of adhering to security practices during the project. The Adherence construct is measured in terms of the frequency and prevalence of security practice use by the team, and in terms of the vulnerability source to which the practice is applied. To measure frequency, we count instances of practice use in team documentation and communications (e.g. project documentation, emails, commit messages, bug tracking issues), and normalize the count to the total total team effort in the units counted. For example, if there are 10 references in emails to penetration testing and there are 100 emails, the ratio is 0.1.  To measure prevalence, we compute the ratio of team members who apply the practice to total team members. We measure adherence for each security practice to enable comparison of the relative values of the various security practices to which a team adheres. To measure source, we record the vulnerability source (Specification, Coding, Testing, Operations) to which the practice is applied. We conjecture that the types of communication the team use indicate influence on different vulnerability sources. For example, we conjecture that practice adherence in commit messages primarily represents security effort during Coding,  and that practice adherence in bug tracking issues is more likely to represent security effort in Testing and/or Operations. 

\subsection{Putting it all together; Linking SPEF to the Model}

Figure X-marks-the-full-model-spot presents the model, annotated with each SPEF data element in its hypothesized relationship to the model. 
In this section, we build three graphical models of the hypothesized relationships described above, representing three levels of measurement detail.






We select projects for this study based upon meeting the following criteria:
\begin{itemize}
	
	\item Available records of software security vulnerabilities
	\item Version control system access, providing both project source code and the history of changes to the code over time
	\item Bug tracker system access, providing records of vulnerability and defect discovery and resolution
	\item Project documentation, providing access to information about the project’s development process and practices
\end{itemize}


We applied the Boa [44] language infrastructure to the full September 2015 Github dataset ro collect project summaries for 68,035 Java language projects We operationalize practice adherence by counting references to each practice’s keywords in commit messages. 

To identify a practice, we adapt Ernst and Mylopoulos’s [28] technique of labeling events in project history, though we measure security practice use rather than requirements discussions, and we use the SP-EF keywords, listed in Table 1, rather than drawing from the ISO 9126 quality model. 

We developed a Boa job  that scans Boa datasets for projects, and produces a single-line summary for each project, including the project name, url, Source Lines of Code (SLOC), number of developers (Devs), references to ‘CVE-‘ records, and keyword counts for each of the security practices and the ‘Security-Related’ keywords.  

To exclude toy projects, we limited the results to those projects with two or more developers, and 1000 or more lines of code.  Because Boa only stores source code information for Java projects, our results include only Java projects. We operationalize security outcomes by counting references to CVE’s in commit messages.  We identified 121 unique projects with one or more CVE’s. 

We randomly sampled 121 projects with no CVE’s from the remaining projects to form a data set of 242 projects.
We were not able to obtain all SP-EF measurements from the repositories. 

To compensate, we kept the structural model intact and adapted or removed some elements of the measurement model to account for the data available from our data sources. 

We now list the adaptations. Since all projects are written in Java, we elided language from the model. 

We removed domain, number of identities, CR, IR, and AR because any selection process we could develop to set values for the variables would add noise to the data. Because of the nature of commit logs, we assume that each unique email address associated with the commit is for a developer. We use the number of developers (‘Devs’), as a proxy for team size. As a proxy for number of machines, we used the number of Google  search results for each repository, reasoning that the search result counts – how widely known the project is - is proportional to how widely used the project is We use counts of CVE references as a proxy for total public vulnerabilities (pre- and post-release), and we calculate VDensity. We do not have a breakdown of when vulnerabilities were discovered for each project, and so we do not measure Pre-Release Vulnerabilities, Post-Release Vulnerabilities, and we do not calculate their ratio, VRE. We use the CVE count and VDensity as our measures of security outcomes. 


\item Estimate number of observations required
Count the edges between each structural model and measurement model variable, multiply by 2 to account for both the variable's parameter estimate and its error term, and by 10 to account for the number of observations required to estimate each parameter. If the dataset is smaller than the number of observations required, consider whether a dataset of sufficent size can be bootstrapped.


\subsubsection{Specification}
For causal hypotheses, we used the hypotheses for each construct in Section \ref{sec:model} to create edges in the model specifying each element’s effect on security risk, practice adherence, and security outcomes.  We model security risk, security effort, and security outcomes as latent variables, explanatory entities that reflect a continuum that is not directly observable~\cite{kline2015principles}. In SEM, latent variables are represented in diagrams as ovals or circles.

\subsubsection{Identification}
In identification, the specified model is checked against statistical theory for whether all of the models parameters can be estimated. Identification is analogous to checking whether a set of equations is solvable given their structure and the data at hand.  Our construct model contains no feedback loops making it both recursive, and, therefore, identified, by Rule 7.1 of Kline ~\cite{kline2015principles}.  



The Common Criteria (CC) ~cite{commoncriteria2012introduction} model for security concepts can be summarized in the following five statements:
\begin{itemize}
	\item Owners value Assets
	\item Owners wish to minimize Risk to Assets
	\item Owners impose Countermeasures to reduce Risk
	\item Threat agents wish to abuse or damage Assets
	\item Threat agents give rise to Threats to Assets that increase Risk
\end{itemize}

We adapt the CC model: we do not model actors, e.g. Owners and Threat agents, and we augment the model with measures of the constructs. Removing actors from consideration allows the model to be evaluated independent of specific accountability and simplifies the model. Risk to Assets, Threats to Assets increase Risk, Countermeasures decrease Risk. Translating to the terms used in our measurement framework, Risk is described in terms of Outcome Measures and Context Factors, Assets are described in terms of Context Factors, and Countermeasures are described in terms of Practice Adherence.

We define a set of vulnerability sources, drawn from our definition:  
\begin{itemize}
	\item \textbf{Specification} - All activities, e.g. planning, preparation, requirements definition, and design that precede, chronologically or conceptually, coding. 
	\item \textbf{Code} - The development and maintenance of software features.
	\item \textbf{Test} - The quality assurance activities applied to developed software before it is released.
	\item \textbf{Operations} - The configuration and use of the released software, and feedback from the software's users to the development organization.
\end{itemize} 

The vulnerability sources may also be viewed chronologically, as development phases. Boehm ~\cite{boehm1981economics} reports that the earlier in the development process a defect is resolved, the cheaper it is to resolve. We can use data collected according to this scheme to analyze whether vulnerabilities follow the same cost pattern as defects.  While many projects will have a more detailed set of development process sources (phases), we conjecture that these phases are present in the majority of software development projects.


For each practice, we used the search strings listed in the ‘Keywords’ column of Table 1, and recorded each instance of practice use.  We collected data by two means:
\begin{itemize}
	\item We used a script, available from the SP-EF website, to iterate over each commit, issue, and email, and generate security practice event classifications.
	\item The first author manually examined each project for security practice use and generated security practice event classifications. Additional raters classified a randomly selected pool of issues, and we compared their results to the automated classifications.
\end{itemize}

For each artifact we identified, we recorded the document name, URL, age, and made note of any available change history, e.g. wiki page change records. We manually classified pre- and post-release vulnerabilities based on a study of who reported the vulnerability and whether they were identifiable as a project member.

Don't forget about the case_* input files

\subsection{OpenSSL Case Study}
\label{sec:case_openssl}

OpenSSL~\footnote{\url{https://www.openssl.org/}} is an open source toolkit for implementing communications over Transport Layer Security (TLS) and (SSL), used, among many uses, by the Apache and nginx web servers to manage encrypted web sessions.   

OpenSSL provides an open defect repository and version control system, both based on a public Github repository~\footnote{\url{https://github.com/phpmyadmin/phpmyadmin}} The project issues security advisories and releases for critical security issues~\footnote{\url{https://www.phpmyadmin.net/security/}}. 

In April 2014, two groups of security researchers reported that the `Catastrophic'~\footnote{\url{https://www.schneier.com/blog/archives/2014/04/heartbleed.html}} Heartbleed~\footnote{\url{http://heartbleed.com/}} vulnerability was discovered in OpenSSL. [Blurb about OpenSSL investment from the Linux Foundation...] The OpenSSL case offers an opportunity for us to investigate whether SP-EF`s practice adherence measures will show evidence of security practice use where we know in advance that such changes exist.

We cloned the OpenSSL github repo~\footnote{\url{https://github.com/openssl/openssl}}, representing the source code and changes made during the history of the projects. We applied CVSAnaly from the MetricsGrimoire tool set reported on by Gonzalez-Barahona~\cite{barahona2015metrics} to process and summarize the github data, and then analyzed and reported on the data using R~\footnote{\url{https://www.r-project.org/}}. We extracted email data from downloaded developer email archives. We extracted defect and vulnerability data from OpenSSL’s github repository~\footnote{\url{https://github.com/openssl/openssl/issues}}, and checked it against the project’s CVE records~\footnote{\url{https://www.cvedetails.com/vendor/383/Openssl-Openssl.html}}.

\subsection{Context Factors}
The context factors SLOC, Churn, and Developers are based on github data, while Number of Machines is conservatively estimated as the total of Apache and nginx servers reported by the monthly Netcraft~\footnote{\url{https://news.netcraft.com/archives/2016/03/18}, \url{https://news.netcraft.com/archives/2014/03/03}} surveys for March 2014 and March 2016. We, subjectively, rate OpenSSL’s Confidentiality, Integrity, and Availability Requirements as High, because the software transmits encrypted data representing millions of sensitive (e.g. web purchases) communications daily.

\subsubsection{Practice Adherence}
In this section, we present our findings on the use of each security practice from two data collection perspectives; qualitative observation and text mining, summarized in Table \ref{tab:paComparisonTable}

\textbf{Apply Data Classification Scheme (ADCS)}
We were not able to identify documented process or artifacts for data classification.

\textbf{Apply Security Requirements (ASR)}
OpenSSL maintains a (incomplete) list~\footnote{\url{https://www.openssl.org/docs/standards.html}} of the Internet Engineering Task Force Request For Comments (RFC) standards that are implemented in the software.  OpenSSL has received FIPS 140~\footnote{\url{https://www.openssl.org/docs/fips.html}} certification.

\textbf{Perform Threat Modeling (PTM)}
We were not able to identify documented process or artifacts around threat modeling.

\textbf{Document Technical Stack (DTS)}
The OpenSSL team maintains a list of dependencies in the INSTALL script for building the software, kept in the github repo. Version number requirements for the dependencies are not listed.

\textbf{Apply Secure Coding Standards (ASCS)}
OpenSSL maintains a coding standards document. ~\footnote{\url{https://www.openssl.org/policies/codingstyle.html}}. We conjecture that manual code reviews of changes to the software are the mechanism for enforcement of the standard.

\textbf{Apply Security Tooling (AST)}
At the time of Heartbleed, no testing tools were incorporated into the offical build for release, although many tool vendors ran their tools on OpenSSL and provided feedback to the project. 

\textbf{Perform Security Testing (PST)}
The OpenSSL github repository contains a test suite for the software. The team began running the test suite on every change, continuous integration, in October 2015~\footnote{\url{https://mta.openssl.org/pipermail/openssl-dev/2015-October/003251.html}}.

\textbf{Perform Penetration Testing (PPT)}
External security researchers conduct penetration testing. Tool vendors provide the project with reports from static and dynamic analyzers, and fuzzing tools. 

\textbf{Perform Security Review (PSR)}
An experienced, full-time project member reviews each change made. In changes after Heartbleed, a 'Reviewed-By' tag was added to each commit message, along with a protocol for additional review of critical changes.

\textbf{Publish Operations Guide (POG)}
The project team maintains a set of documents, manual pages, and a FAQ~\footnote{\url{https://www.openssl.org/docs/}}.

\textbf{Track Vulnerabilities (TV)}
The project team publishes procedures for raising security concerns, and maintains a security policy for handling security-related matters. 

\textbf{Improve Development Process (IDP)}
The projec team refines the development process over time.

\textbf{Perform Security Training (PST)} 
We were not able to identify documented process or artifacts around security training for the project team.


%\input{PracticeAdherenceData_OpenSSL}

\subsubsection{Outcome Measures}

\begin{table*}
	\begin{center}	
		\caption{Case Study Context Factors: OpenSSL and phpMyAdmin}
		\begin{tiny}	
			\begin{tabular}{|l|l|l||l|l|}
				%\topline
				%\headcol & \multicolumn{2}{c}{OpenSSL} & \multicolumn{2}{c}{phpMyAdmin}\\
				%\headcol & Before & After & Before & After \\	
				%\midline
				Churn & & & & \\
				Confidentiality Requirement & High & High & High  & High \\
				Integrity Requirement & High & High & High  & High \\
				Availability Requirement & High & High & Low  & Low \\
				Dependencies & glibc & glibc & OS, Web Server, MySQL, Browser &  OS, Web Server, MySQL, Browse \\
				Domain & networking, security & networking, security & Database administration & Database administration \\
				Number of Identities & Millions & Millions & Millions & Millions \\
				Language & C & C & PHP, Javascript, HTML SQL & PHP, Javascript, HTML SQL \\
				Number of Machines & 900,000 & 3,700,000 & Millions & Millions \\
				Methodology & & & BDFL & BDFL \\
				Operating System & Unix, Windows, OS X, Open VMS & Unix, Windows, OS X, Open VMS & Unix, Unix \\
				Product Age & 18 years & 18 years & 18 years & 18 years \\
				SLOC & 456,000 & 438,000 & 154,000 & 389,000 \\
				Source Code Availability &  Open Source &  Open Source &  Open Source &  Open Source \\
				Team Location & Distributed  & Distributed  & Distributed  & Distributed \\
				Team Size (Devs,Testers) & 15  & 15 & 9 & 9 \\
				%\bottomline
			\end{tabular}
			
			\label{tab:CAO_SCTO_Table}
		\end{tiny}
	\end{center}
\end{table*}

\begin{table*}
	\begin{center}	
		\caption{phpMyAdmin Practice Adherence Comparison Table}
		\begin{tiny}
			
			\begin{tabular}{|l|rrr||rrr|}
				%\topline
				%\headcol & \multicolumn{3}{c}{Before} & \multicolumn{3}{c}{After}\\
				%\headcol  & Observed & Oracle & Mined  & Observed & Oracle & Mined \\ 
				%\headcol  SP-EF Security Practice  & Frequency & Count & Count & Frequency & Count & Count\\
				%\midline	
				Apply Data Classification Scheme & &  &  & N/A & & 2\\
				Apply Security Requirements  & &  & 25  & Weekly & 6 & 50\\
				Perform Threat Modeling &  &  & 2  & Weekly & & 9\\
				Document Technical Stack  &  &  & 123  & Monthly & & 496\\
				Apply Secure Coding Standards  &  &  & 90  & Daily & & 754\\
				Apply Security Tooling  &  &  & 19 & Daily & & 129\\
				Perform Security Testing &  &  & 48  & Weekly & 285 & 521\\
				Perform Penetration Testing  &  &  & 3  & Monthly & & 0\\
				Perform Security Review  &  &  & 2  & Daily & 7 & 40\\
				Publish Operations Guide &  &  & 101 & Monthly & & 388\\
				Track Vulnerabilities & & 77 & 17  & Weekly & 166 & 498\\
				Improve Development Process & & 44 &  & Monthly & 48 & 4\\
				Perform Security Training &  &  & 25 & N/A & & 196\\
				%\midline
				CVE & 17 & N/A & N/A & 28 & N/A & N/A\\
				Security-Related &  &  & 45 & & & 132 \\
				
				%\bottomline
			\end{tabular}
			
			\label{tab:openssl_paComparisonTable}
		\end{tiny}
	\end{center}
\end{table*}


\begin{table*}
	\begin{center}	
		\caption{OpenSSL Practice Adherence Comparison Table}
		\begin{tiny}
			
			\begin{tabular}{|l|rrr||rrr|}
				%\topline
				%\headcol & \multicolumn{3}{c}{Before} & \multicolumn{3}{c}{After}\\
				%\headcol  & Observed & Oracle & Mined  & Observed & Oracle & Mined \\ 
				%\headcol SP-EF Security Practice  & Frequency & Count & Count & Frequency & Count & Count\\
				%\midline	
				Apply Data Classification Scheme & &  & 0 & N/A & & 2\\
				Apply Security Requirements  & &  & 13 & Weekly & & 67\\
				Perform Threat Modeling &  &  &  & Weekly & & 18\\
				Document Technical Stack  &  & 2 & 114 & Monthly & 6 & 433\\
				Apply Secure Coding Standards  &  &  & 282 & Daily & & 553\\
				Apply Security Tooling  &  & 0 & 12 & Daily & 0 & 65\\
				Perform Security Testing &  & 0  & 212 & Weekly & & 261\\
				Perform Penetration Testing  &  & & 0 & Monthly & & 0\\
				Perform Security Review  & Daily & 0 & 12 & Daily & 2 & 543\\
				Publish Operations Guide &  &  & 81 & Monthly & & 215\\
				Track Vulnerabilities & & 5 & 181 & Weekly & 99 & 197\\
				Improve Development Process & & 0 & 0 & Monthly & 4 & 1\\
				Perform Security Training &  &  & 56 & N/A & 12 & 36\\
				%\midline
				CVE &  &  &  & &  & \\
				Security-Related &  &  & 20 & & & 242 \\
				
				%\bottomline
			\end{tabular}
			
			\label{tab:pma_paComparisonTable}
		\end{tiny}
	\end{center}
\end{table*}

\subsection{phpMyAdmin Case Study}
\label{sec:case_pma}
	
phpMyAdmin is a popular MySql administration tool, with rich documentation including various books~\footnote{\url{https://www.phpmyadmin.net/15-years/}}. phpMyAdmin provides an open defect repository and version control system, both based on a public Github repository~\footnote{\url{https://github.com/phpmyadmin/phpmyadmin}} The project issues security advisories and releases for critical security issues~\footnote{\url{https://www.phpmyadmin.net/security/}}. 
phpMyAdmin has participated in the Google Summer of Code~\footnote{\url{https://developers.google.com/open-source/gsoc/}} annually since 2008. phpMyAdmin`s administrators have introduced documentation and process changes in conjunction with its GSOC participation~\footnote{\url{https://www.phpmyadmin.net/news/2008/4/4/google-summer-of-code-2008-and-phpmyadmin/}}. The phpMyAdmin case offers an opportunity for us to investigate whether SP-EF`s practice adherence measures will show evidence of security practice use where we know in advance that such changes exist.

We cloned the phpmyadmin github repo, representing the source code and changes made during the history of the project, encompassing 2005 source files, and 99121 commits by 955 unique developers. We applied CVSAnaly from the MetricsGrimoire tool set reported on by Gonzalez-Barahona~\cite{barahona2015metrics} to process and summarize the github data, and then analyzed and reported on the data using R~\footnote{\url{https://www.r-project.org/}}. We extracted email data from downloaded developer email archives. We extracted defect and vulnerability data from phpMyAdmin’s github repository and from the security section of its website, and checked it against the project’s CVE records~\footnote{\url{https://www.cvedetails.com/vendor/784/Phpmyadmin.html}}. 

\subsubsection{Context Factors}

The context factors SLOC, Churn, and Developers are based on version control system data, while Number of Machines is estimated based on 20\% of 200,000 monthly downloads reported in September 2013, projected linearly from the project start date. We, subjectively, rate phpMyAdmin’s Confidentiality Requirement and Integrity Requirement as High, because the software supports administrator creation, editing, and deletion of MySql database schemas and data. We rate phpMyAdmin’s Availability Requirement as Low, because the software is a optional, and is not essential to MySql database administration.

\subsubsection{Practice Adherence}

In this section, we present our findings on the use of each security practice from two data collection perspectives; qualitative observation, and text mining, summarized in Table \ref{tab:paComparisonTable}.

\subsubsection{Researcher Observation}

In this section, we review security practice usage results obtained from researcher assessment of phpMyAdmin. We present observations organized by security practice, and include links to evidence, where available. Italicized quotes are from phpMyAdmin project communications.

\textbf{Apply Data Classification Scheme (ADCS)}
We were not able to identify documented process or artifacts for data classification.

\textbf{Apply Security Requirements (ASR)}
The team considers security on a case-by-case basis as code is written. The two main administrators have been with the project since shortly after its inception, and they monitor features, and issues for security as well as other quality attributes.

\textbf{Perform Threat Modeling (PTM)}
We were not able to identify documented process or artifacts around threat modeling. 

\textbf{Document Technical Stack (DTS)}
The phpMyAdmin team documents their technical stack via a wiki page.

\textbf{Apply Secure Coding Standards (ASCS)}
Over time, the phpMyAdmin project has adopted coding standards, and security coding standards. August 2010: \textit{Please stick with PEAR coding style and please try to keep your code as simple as possible: beginners are using phpMyAdmin as an example application.}  March 2012: \textit{Here are some guidelines on how to avoid security issues that could lead to security bugs}.

\textbf{Apply Security Tooling (AST)}
Over time, the project has added tooling to support its standards and build process. May 2012:\textit{To verify coding style, you can install PHP CodeSniffer}  Sep 2013: \textit{I\`ve set up coveralls.io coverage reports}. Dec 2013: \textit{(r.e. PHP code checking tool, scrutinizer) What is this good for?}. May 2014 \textit{Let\`s make scrutinizer happy ;)}. 

\textbf{Perform Security Testing (PST)}
One of the project\`s leaders confirmed~\footnote{via private email communication} that external security consultants (researchers) are the primary finders of vulnerabilities. However, the team began developing an automated test suite in 2009, and continues to maintain and add to it. May 2001:  \textit{It would maybe nice to have a kind of `test suite'}. Jul 2009: \textit{I\`ve set up a cronjob which would run the test suite e.g. every day}. Nov 2014 \textit{'We use several test suites to help ensure quality code. All new code should be covered by test cases to ensure proper [Unit Testing]'}

\textbf{Perform Penetration Testing (PPT)}
External security researchers conduct penetration testing. 

\textbf{Perform Security Review (PSR)}
The two main administrators monitor the build process and its associated tool reports for security as well as other quality attributes.

\textbf{Publish Operations Guide (POG)}
The team maintains installation, configuration, and administration documentation for phpMyAdmin, including security-related material%\footnote{\url{http://docs.phpmyadmin.net/en/latest/faq.html#security}}. 

\textbf{Track Vulnerabilities (TV)}
The phpMyAdmin project maintains a security mailing address, and a list of vulnerabilities resolved. The team tracks and resolves vulnerabilities as they are reported.

\textbf{Improve Development Process (IDP)} In conjunction with joining the GSOC in 2008, and managing the additional help they received from the sponsored developers, phpMyAdmin administrators added or extended standards, tooling, and test suites to the development process.

\textbf{Perform Security Training (PSTR)} 
While we did not observe direct evidence of training, e.g. tutorials or courses, the project has brought developers in successfully through its mentoring program. 

\subsubsection{Text Mining}
A display of counts per Project Month for each Issue containing one or more of a practice’s keywords is presented in Figure 2.





\subsubsection{Outcome Measures}
We present phpMyAdmin Vdensity and VRE for 2008-2014 in Figure 3.  




