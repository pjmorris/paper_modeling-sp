\section{Discussion}
\label{sec:discussion}

In this section, we discuss the implications of our case study results.

Our analysis of the NVD data supported our hypotheses that higher Asset Values and higher Software Risk are associated with worse Security Outcomes, as measured by CVSS metrics and project CVE counts. Our time-based metric for practice Adherence was positively associated with Software Risk, contrary to our hypothesis. We, further, found Access Complexity to not fit within our current model, suggesting further understanding and, perhaps, model revision is required. Our data suggest that the network attack vector has become predominant in the vulnerabilities reported in the NVD, but that access control attacks are growing more difficult for attackers, as measured by the CVSS Access Vector and Access Complexity metrics. 

\begin{itemize}
	\item Revise this once you have the actual NVD discussion rewritten. The NVD data offers per-manufacturer, per-project overview of reporter interest in software vulnerabilities, evidence that per-vulnerability severity is decreasing, and confirmation that the network is an important attack vector.
	\item The CII data offers evidence that SLOC and team size are correlated with manifest vulnerabilities. 
	\item Combined NVD-CII data supports better model fit, and offers theoretical support for X, Y, Z elements of the SP-EF measurement model. 
	\item Stepping back from the specifics of the case studies, SEM offers means of assessing the relative importance of the measurements taken for software security assessement.
	\item Asset Value, Software Risk, Adherence, and Outcomes appear to be a reasonable place to start for organizing studies of the factors behind software security success and failure. 
\end{itemize}

