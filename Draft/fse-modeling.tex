\documentclass[sigconf,anonymous]{acmart}

\usepackage{booktabs} % For formal tables
\usepackage{enumitem}
\usepackage{natbib}
\usepackage{mathtools}
\usepackage{amssymb}
\usepackage{amsthm}
\citestyle{acmnumeric}


% Copyright
%\setcopyright{none}
%\setcopyright{acmcopyright}
%\setcopyright{acmlicensed}
\setcopyright{rightsretained}
%\setcopyright{usgov}
%\setcopyright{usgovmixed}
%\setcopyright{cagov}
%\setcopyright{cagovmixed}

% CFP: http://esec-fse17.uni-paderborn.de/call_for_papers.php
% Template: http://www.acm.org/publications/proceedings-template

% DOI
\acmDOI{10.475/123_4}

% ISBN
\acmISBN{123-4567-24-567/08/06}

%Conference
\acmConference[ESEC/FSE'17]{ACM conference}{September 2017}{El
	Paderborn, Germany} 
\acmYear{2017}
\copyrightyear{2017}

\acmPrice{15.00}


\begin{document}
%\title{Got To Go}

\title{Quantifying Security Context Factors in Software Development}


\author{Patrick Morrison}
\affiliation{%
	\institution{Department of Computer Science}
	\streetaddress{North Carolina State University}
	\city{Raleigh}
	\state{North Carolina} 
	%\postcode{43017-6221}
}
\email{pjmorris@ncsu.edu}

% Asked to be left off of paper, pending closer review
%\author{Jonathan W. Stallings}
%\affiliation{%
%	\institution{Department of Statistics}
%	\streetaddress{North Carolina State University}
%	\city{Raleigh}
%	\state{North Carolina} 
%	%\postcode{43017-6221}
%}
%\email{jwstallings@ncsu.edu}

\author{Laurie Williams}

\affiliation{%
	\institution{Department of Computer Science}
	\streetaddress{North Carolina State University}
	\city{Raleigh}
	\state{North Carolina} 
	%\postcode{43017-6221}
}
\email{williams@csc.ncsu.edu}

	
\begin{abstract}
\label{sec:abstract}
Context: Researchers and practitioners have proposed and evaluated many practices for measuring and improving software security during software development. However, assessing the effectiveness of security practices requires measuring not only the practices and their effects on software, but the software's characteristics and the context in which the software is used. 

Goal:  \textit{The goal of this research is to support researchers and practitioners in measuring the effect of security practice use on software vulnerabilities by building and evaluating an explanatory model for software development security practice use.} 

Method: In this work, we develop a model hypothesizing relationships between security outcomes, adherence to security practices, software context factors, and environmental context factors.  We evaluate the model quantitatively, using two existing datasets: The National Vulnerability Database (NVD), containing 79,533 vulnerabilities, and the Core Infrastructure Initiative (CII) census dataset of 428 security-critical projects.  

Results: As measured in our data, both environmental context factors and software characteristics had strong correlations with security outcomes. The NVD case study results agreed with previous research suggesting that software security has improved over the last decade, as measured by decreasing CVSS scores and increasing access complexity required of attackers. The CII case study results agreed with previous research suggesting that team size and effort, and code size, are among the most significant influences on vulnerabilities. 
Conclusion: The developed model and results from the two case studies offer a guide to the metrics suitable for prioritizing management of software context factors and security outcomes. 


\end{abstract}

%\ccsdesc[500]{Keywords got to go~Redundancy}
%\ccsdesc[300]{Computer systems %organization~Redundancy}
%\ccsdesc{Computer systems organization~Robotics}
%\ccsdesc[100]{Networks~Network reliability}

% We no longer use \terms command
%\terms{Theory}

%\keywords{ACM proceedings, \LaTeX, text tagging}

\maketitle

\section{Introduction}
\label{sec:intro}
In February 2017, security researchers discovered a programming bug in an HTML parser at Cloudflare\footnote{https://blog.cloudflare.com/incident-report-on-memory-leak-caused-by-cloudflare-parser-bug/}. The parser was sending decrypted encryption keys and passwords along with ordinary web requests, resulting in unencrypted sensitive data being cached across the Internet. While the programming bug, now known as `Cloudbleed'~\footnote{https://blog.cloudflare.com/quantifying-the-impact-of-cloudbleed}, was on a single line of code, the bug had widespread effects because of the code's context, running on many servers, and sending sensitive results to many clients. The example of Cloudbleed suggests that software development teams must consider software's environment and data as well as the technical characteristics of their software when evaluating their software's security. 

The IEEE defines security~\cite{ieee1990glossary} as 'all aspects related to defining, achieving, and maintaining confidentiality, integrity, availability, non-repudiation, accountability, authenticity, and reliability of a system.' Several lists of practices for specifying how a development team defines, achieves, and maintains the various security properties in the software the team produces have been published. Example lists include the Building Security in Maturity Model ~\cite{mcgraw2006software}
(BSIMM), the Microsoft Security Development Lifecycle ~\cite{howard2009security} (SDL), the Software Assurance Forum for Excellence in Code~\cite{simpson2013fundamental} (SAFECode), and the Open Web Application Security Project (OWASP) Software Security Assurance Process ~\cite{martinez2014ssap} (SSAP). While defining how a system will support the various security properties lies within the control of the development team, achieving and maintaining security also depends on factors outside of the development team's direct control. Two pieces of software may have identical technical characteristics (e.g. same language, same size, same complexity) but different security outcomes as a consequence of the uses to which the software is put. For example, between a game level editor and a database system with identical technical characteristics, the database system managing credit card data may be more prone to attack, and to manifest vulnerabilities~\footnote{Following Krsul~\cite{krsul1998software} and Ozment~\cite{ozment2007vulnerability}, we define a vulnerability as “an instance of a mistake in the specification, development, or configuration of software such that its execution can violate the explicit or implicit security policy.}, than the game level editor managing custom game levels. 

Multiple software technical characteristics contribute to software vulnerabilities, for example, code size ~\cite{alhazmi2007measuring}, code churn ~\cite{shin2011evaluating,meneely2013when}, and language ~\cite{ray2014a}. Similarly, multiple software usage characteristics contribute to software vulnerabilities, for example, access to sensitive data~\footnote{http://heartbleed.com/}, management of financial information~\cite{harris2014for}, and the presence of a piece of software on large numbers of machines~\footnote{http://www.cert.org/historical/advisories/CA-2001-19.cfm} have all been associated with software vulnerabilities. Finally, the practices a development team chooses to apply, and the degree to which they adhere to those practices, affects the degree of security provided by their software. As an example of a specific practice, teams may apply analysis tools like static analyzers and fuzzers to detecting code flaws before release. However, considerations like whether the team has been trained to use the tools, whether the tools are regularly or occasionally applied, and whether the tool guidance is required to be applied to the code will affect how effective the tools are in achieving security properties.  Two teams that use the same set of tools may differ in outcomes due to their degree of adherence to the practice of applying security tooling to their software. Development teams and researchers could benefit from a comprehensive picture of the factors underlying software security concerns.

\textit{The goal of this research is to support researchers and practitioners in measuring the effect of software technical characteristics, software usage characteristics, and security practice adherence on software security outcomes by building and evaluating an explanatory model using structural equation modeling.}

We propose and evaluate a model, the \ModelName (\ModelAbbr), for quantifying security practice use and outcomes during software development. The four constructs of our model are:
\begin{enumerate}
	\item Software Risk - measures of technical characteristics of the software that have been shown to be associated with vulnerabilities and defects;	
	\item Asset Impact - measures of software usage characteristics associated with the value an attacker will find in conducting a successful attack;
	\item Practice Adherence (Adherence) - measures of the development team's security assurance efforts; 
	\item Security Outcomes (Outcomes) - measures of security-related indications associated with a piece of software over the course of the software's life cycle.
\end{enumerate}

We hypothesize that the four constructs are related as follows:
\begin{itemize}
	\item \textbf{H1} Asset Impact is associated with negative Security Outcomes
	\item \textbf{H2} Software Risk is associated with negative Security Outcomes
	\item \textbf{H3} Practice Adherence is associated with Software Risk 	
\end{itemize}
	
To evaluate the model and our hypotheses, we conduct two quantitative studies of the construct relationships, applying data from the National Vulnerability Database (NVD) and the Core Infrastructure Initiative (CII) Census to test our hypotheses.
  
Our contributions include:
\begin{itemize}
\item An explanatory model for how software development security practice use and security risk affect security outcomes;
\item Two case study evaluations on how our model fits collected data from software projects, and
\item a set of guidelines for applying our model to new software projects. 
\end{itemize}
The remainder of this paper is organized as follows:  Section \ref{sec:background} discusses background and related work. Section \ref{sec:model} presents an overview of our model and its underlying hypotheses. Section \ref{sec:methodology} presents our study methodology. Section \ref{sec:evaluation} presents the case studies and their results. Section \ref{sec:discussion} discuses the measurement results. Section \ref{sec:limitations} presents our study limitations. Section \ref{sec:conclusion} presents our conclusion.
\section{Background}
\label{sec:background}
In this section we provide work related to structural equation modeling, security practices, and measurement frameworks.

\subsection{Structural Equation Modeling}
Structural Equation Modeling (SEM) is a family of statistical techniques for testing theories by specifying models that represent predictions of that theory among plausible constructs, measured with appropriate observed variables~\cite{kline2015principles}. Suhr provides a concise online tutorial~\footnote{http://www.lexjansen.com/wuss/2006/tutorials/TUT-Suhr.pdf}. While most traditional statistical methods emphasize the modeling of individual observations, SEM emphasizes the covariances of the observed variables~\cite{hildreth2013residual}. In SEM, a model is posited that specifies the relationships among variables, resulting in systems of linear equations such that the relationships between all variables are linear (or transformably linear). In these linear equations, variables are linked by structural parameters indicating the expected (modeled) relationships between variables. Based on these equations, the population covariance matrix of the observed variables can be represented as a function of the model parameters. The population covariance matrix is derived from datasets representing a sample drawn from the population, and `estimator' algorithms are applied to the model and dataset to solve for the model parameter estimates. Hildreth ~\cite{hildreth2013residual}, Chapter 2, gives a clear presentation of the linear algebra involved in specifying and solving SEM models. 

We rely, in particular, on two features of SEM to support our investigation: the distinction between observed variables and latent variables, and the distinction between structural and measurement models.  

Observed variables are the data researchers collect to study a construct of interest. However, not every quantity we wish to measure can be measured directly. For example, in psychological studies of `general intelligence', \textit{g}, researchers examine correlations between various and multiple observed test scores and the notion, or `construct' of general intelligence. Latent variables are a tool for analyzing combinations of observed variables that correspond to hypothetical constructs, which are presumed to reflect a continuum that is not directly observable ~\cite{kline2015principles}. Researchers commonly investigate the effects of unobservables like intelligence by statistically relating covariation between observed variables to latent variables~\cite{borsboom2003theoretical}.  

%One or more observed variables may be modeled as either causing (formative) or being caused by (reflective) a latent variable. 
We adopt latent variables as a suitable tool for relating observed data to the constructs we study, for example Asset Impact, and Software Risk.

The distinction between structural and measurement models parallels the distinction between latent variables and observed variables. Structural models are representations of the latent variables and the relationships between them. Measurement models are representations of observed variables and the relationships between them.  A measurement model may be linked to a structural model to indicate relationships between observed and latent variables. Repeated studies with a single structural-measurement model combination contribute evidence for accepting or rejecting the combined models. A single structural model may be associated, serially, with a variety of measurement models, reflecting a variety of approaches to measuring a theoretical construct.  Observations of similar trends using different data sets and measurement approaches lends strength to the underlying theory~\cite{basili1999building,wohlin2000experimentation}. 

A number of software packages, including sem~\footnote{https://cran.r-project.org/web/packages/sem/index.html}, MPlus~\footnote{https://www.statmodel.com/}, and lavaan~\cite{roseel2012lavaan}, provide tools for specifying SEM models and a variety of estimation algorithms to match the characteristics of models and datasets. 

SEM studies are organized around six steps (which may be iterated over, as called for by the needs of the study): 
\begin{itemize}
\item \textit{model specification}, researchers express the hypothesized relationships between observed variables and latent variables, typically in the form of a graphical model. Each edge in the graph represents a parameter to be estimated, indicating the strength of the relationship between the nodes connected by the edge,

\item \textit{identification}, the specified model is checked against statistical theory for whether all of the model’s parameters can be estimated given the model's structure. Identification is analogous to checking whether a set of equations is solvable given their structure and the data at hand. If the original model cannot be identified, it must be revised in light of both statistical theory and the theory the researcher is expressing in the model,

\item \textit{data selection}, where data for each of the model’s observed variables is chosen. Kline ~\cite{kline2015principles} states that SEM is a large-sample technique, reporting median sample size in the literature of 200 cases, starting with a minimum of 10 observations per parameter to be estimated.  Factors that drive the need for a large sample include the size and complexity of the model to be estimated, non-normally distributed data, categorical data, and imprecisely measured data,

\item \textit{data collection}, where the selected data is obtained from the study's sources,

\item \textit{estimation}, the observed data and the model are checked for fit.  If appropriate fit is achieved, the parameter estimates can be interpreted for implications of the theorized relationships and the observed data. If appropriate fit is not achieved, the list of model changes developed during specification should be considered in re-specifying the model, and

\item \textit{reporting results}, the model, parameter estimates,  fit measures, and changes made for re-specification should be included in the report.

Examples of SEM use in software engineering and information technology include Capra et al.~\cite{capra2008empirical}, Wallace and Sheetz~\cite{wallace2014adoption}, and Gopal et al.~\cite{gopal2005impact}.
\end{itemize}

\subsection{Measurement Frameworks}
Measurement frameworks offer a foundation for aggregating study results by standardizing the collection of empirical evidence, enabling comparison of results across projects. Previous researchers~\cite{kitchenham1999towards,williams2004toward} have suggested four reasons for measurement frameworks:
\begin{itemize}
	\item To allow researchers to provide a context within which specific questions can be investigated;
	\item To help researchers understand and resolve contradictory results observed in empirical studies;
	\item to provide researchers a standard framework to assist in data collection and reporting of empirical studies in such a manner that they can be classified, understood, and replicated and to help industrial adoption of research results; and
	\item  to provide researchers a framework for categorizing empirical studies into a body of knowledge.
\end{itemize}

Williams et al.~\cite{williams2004toward} defined a measurement framework for evaluating the use of Extreme Programming (XP), XP-EF. XP-EF contains context factors to capture internal project-related variables; adherence metrics to capture XP practice use; and outcome measures to capture external project results (e.g. quality). Modeled after XP-EF, we have developed SP-EF, a measurement framework for software development security practice use. We defined a similarly structured set of measures for recording context factors, practice adherence metrics, and outcome measures, related to the use of security practices in software development.
Rudolph and Schwartz~\cite{rudolph2012critical} conducted a systematic literature review, and Morrison et al.~\cite{morrison2014mapping} produced a technical report on security metrics. We chose the context factors and outcome measures aligned with the findings of these surveys.

The Common Vulnerability Scoring System~\cite{mell2006common} (CVSS) is a framework for communicating the characteristics of vulnerabilities in information technology (IT). Our focus is on the software development process and product rather than the individual vulnerability, however we adopt the Confidentiality Requirement, Integrity Requirement, and Availability Requirement elements of CVSS as context factors.
\section{The \ModelName}
\label{sec:model}
% Sanders~\cite{sanders2009security} argues that existing security metrics should be integrated to provide a comprehensive, quantified view of systems through their lifecycle. 
We propose the, \ModelAbbr model of the factors influencing security outcomes in software development to enable assessment of how varying security practice use affects those outcomes. 
 
 The Common Criteria (CC) introduction~\cite{common2012common} lists a set of \textbf{security concepts} (concepts bolded) and relationships, summarized as follows:
 \begin{itemize}
 	\item  Owners value \textbf{Assets}, and  impose \textbf{Countermeasures} to reduce \textbf{Risk}.
 	\item Threat Agents give rise to \textbf{Threats} that affect \textbf{Assets} and increase \textbf{Risk}.
 \end{itemize}
 
 Considering the CC concepts in the context of software development and use, we propose a model to capture the distinction between the risk to Assets caused by Threats (Asset Impact) and the risk caused by the software that manages the Assets (Software Risk)  when measuring practice Adherence's effect on security Outcomes. We expect our model to be useful in assessing the relationship of software quality factors and software usage factors with security outcomes, and in assessing the effect of security practice adherence on software quality. Given the complexity of software security, we expect that this initial model will be mis-specified to some degree, and we will use our data collection to evaluate the model and to evaluate possible refinements.

We have theorized relationships between measurement variables and each construct in our model. For example, we theorize that the already mentioned code size and code churn metrics influence Software Risk. We present our list of measurements in Section \ref{sec:model_measurement}.

\subsection{Structural Model Constructs}
 \label{sec:model_structual}
To make claims about how security practices affect security outcomes, we need to account for other influences on security outcomes. In this section, we define the Software Risk, Asset Impact, Adherence, and Outcomes constructs, and the relationships we expect between each construct. 

\subsubsection{Software Risk}
Software Risk (CC \textbf{Risk} potential) represents the characteristics of the software under the control of the development team that are associated with defects and vulnerabilities. In the case of software vulnerabilities, for example, high code churn and defect-prone languages have been correlated with vulnerabilities.

\subsubsection{Asset Impact}
Asset Impact (CC \textbf{Assets}) represents the characteristics of the software's purpose and usage context that are associated with attacker interest. One component of attacker interest is the value of the assets managed by the software. For example, we hypothesize that software tracking valuable or sensitive data, such as Personally Identifiable Information (PII) or credit card data is more likely to be attacked than software tracking, say, baseball scores. As another example, attacker control over a machine enables the machine's participation in a botnet, making the number of machines on which a piece of software runs a consideration in evaluating the software's Asset Impact.

\subsubsection{Adherence}
\label{sec:model_contruct_adherence}
Adherence (CC \textbf{Countermeasures}) represents the efforts the team takes to prevent and discover vulnerabilities. We adapt an IEEE definition of practice~\cite{ieee1990glossary} `a specific type of professional or management activity that contributes to 
the execution of a process and that may employ one or more techniques and tools' to define a software development security practice to be an action a software development team member takes to prevent, identify, or resolve a vulnerability, possibly guided by a tool or reference. We measure the Adherence construct in terms of the frequency  of security practice use by the team.  
% emails - spec
% commit messages - code
% tests - test
% issues - ops
% documentation - spec

\subsubsection{Outcomes}
\label{sec:model_contruct_outcome}
The Outcomes (CC \textbf{Risk} realized) construct represents indications of the failure or achievement of security associated with a piece of software over the course of the software's life cycle.

 At present, counting vulnerabilities, is the most common means of measuring security in software~\cite{morrison2014mapping}. We distinguish between undiscovered (`latent') and discovered (`manifest') vulnerabilities. We, further, distinguish between vulnerabilities identified before the software is released (`Pre-release'), and vulnerabilities identified after the software is released (`Post-release'). 
We measure the Outcomes construct in terms of manifest vulnerabilities and the timing of their discovery and resolution. Low total values for manifest vulnerabilities are preferable, and a high proportion of vulnerabilities discovered pre-release rather than post-release is also preferable. Low vulnerability counts must be interpreted with caution, as they may reflect low use of the system, or low vulnerability discovery effort by the development team, rather than the absence of latent vulnerabilities. 

\subsection{Structural Model Relationships}
We hypothesize that the four constructs are related as follows:
\begin{itemize}
	\item \textbf{H1} Asset Impact is associated with negative Security Outcomes
	\item \textbf{H2} Software Risk is associated with negative Security Outcomes
	\item \textbf{H3} Practice Adherence is associated with Software Risk 	
\end{itemize}

For example, a carefully-written piece of widely-used software that manages financial data (high Asset Impact, low Software Risk, e.g. Bitcoin with 22 CVEs~\footnote{\url{https://www.cvedetails.com/vulnerability-list/vendor_id-12094/Bitcoin.html}}) might have worse Outcomes than a less well written baseball scores program used by a few hundred users (low Asset Impact, high Software Risk, no CVEs reported) because attackers expend more effort on the software managing financial data. We would expect Adherence to be correlated with Asset Impact, as teams adopted security practices in proportion to the security needs of their software, its usage context, and their users. In practice, for example in the case of the CloudBleed example from the introduction of this paper, users (especially attackers) sometimes surprise development teams in the uses of their software, unexpectedly increasing the software's Asset Impact out of proportion to the team's Adherence. 


Figure~\ref{fig:model_constructs} depicts the constructs and their relationships.  Each circle in the figure represents a construct, modeled as a `latent variable'. We model the constructs using latent variables to indicate that our measures for each construct are aggregates of the measurement (observed) variables ~\cite{kline2015principles,borsboom2008latent}. 

Directed edges from circles to other circles, for example the arrow from Asset Impact to Outcomes in Figure \ref{fig:model_constructs}, represent a source latent variable's effect on the target latent variable, indicated by the sign and magnitude of the parameter estimate on the edge between the latent variables. The magnitudes of the parameter estimates are affected by the scales of the measurement variables and by any transformations applied to the measurement variables. These attributes limit the interpretation of parameter estimates to describing that changes in a source latent variable are correlated with changes in a target latent variable in the direction of the parameter estimate's sign and of a magnitude comparable to other parameter estimates. 

Dual arrows between circles/constructs, for example between Adherence and AssetImpact in Figure \ref{fig:model_constructs}, represent covariances between the constructs, implying that a relationship exists, but that the direction of influence is not specified. Dual arrows starting and ending at the same variable indicate residual variance, the amount of variation in the observed variable not explained by the model.

Each square in Figure \ref{fig:model_example_syntax_asmeasuredby} represents a measurement variable associated with each construct. Directed edges (single arrows) from circles to squares, for example from SoftwareRisk to SLOC as shown in Figure \ref{fig:model_example_syntax_asmeasuredby}, represent that a construct `is measured by' a measurement variable relationship. That is to say that the effect of the construct can be measured by the measurement variable. The parameter estimate on these relationships represents the sign and magnitude of the relative contribution of the measurement variable to the construct value. We present the list of measurements for each construct in the measurement guidebook~\footnote{http://pjmorris.github.io/Security-Practices-Evaluation-Framework/guidebook.html}, and we present the measurement variables that we select from each dataset in the case studies. 

\subsection{Measurement Model: Metrics, Variables, and Relationships}
\label{sec:model_measurement}
%~\cite{morrison2014mapping}
%~\cite{morrison2016spefsite}
Through literature review~\cite{morrison2014mapping} and analysis~\cite{morrison2017surveying,morrison2017measuring}, we have developed a set of measurements that we expect to capture security-related constructs for software development. To collect empirical data, we have developed a data collection framework for the measurement model, available online~\footnote{\url{http://pjmorris.github.io/Security-Practices-Evaluation-Framework}}. In Table \ref{tab:model_spef_metrics}, we name each data element, give our hypothesis about its relationship to the structural model construct, and cite a rationale for the data element's presence. 
		
\begin{table*}[!htbp] \centering 
	\caption{Model Metrics and Hypotheses} 
	\label{tab:model_spef_metrics} 
	\begin{scriptsize}
		\begin{tabular}{p{1.75cm}p{1cm}p{1cm}p{6cm}} 
			\\[-1.8ex]\hline 
		\hline \\[-1.8ex] 
		Metric & \multicolumn{1}{c}{Effect} & \multicolumn{1}{c}{Construct} & \multicolumn{1}{c}{Rationale} \\ 
		\hline \\[-1.8ex]  
			Language	& influences &	Software Risk &   Ray et al. ~\cite{ray2014a} and Walden et al. ~\cite{walden2010idea} found small but significant effects of programming language on software quality. Zhang~\cite{zhang2014towards} identifies language as a key context factor. \\
			Operating System	& influences &	Software Risk & \\	
			Domain &	influences &	Software Risk	 & Different risks are associated with different software domains~\cite{williams2004xpef,jones2000software} \\
			Product Age	& increases &	Software Risk & Kaminsky et al.~\cite{kaminsky2011showing} and Morrison et al.~\cite{morrison2015challenges} have found evidence of code age effects on the presence of vulnerabilities. \\
			
			Source Lines of Code (SLOC)	& influences	& Software Risk & Source code size is correlated with vulnerabilities ~\cite{shin2011evaluating}, ~\cite{alhazmi2007measuring}. Zhang~\cite{zhang2014towards} identifies SLOC as a key context factor. \\
			Churn &	increases &	Software Risk  &  Code churn is correlated with vulnerabilities ~\cite{shin2011evaluating}.\\
			Team Size	& influences	& Software Risk & Shin et al. ~\cite{shin2011evaluating} and Zimmermann et al. ~\cite{zimmerman2010searching} found correlations between team size and vulnerabilities. \\			
			\hline \\[-1.8ex] 
			Number of Machines &	increases &	Asset Impact & (Proposed) The market for machine time on botnets suggests that the number of machines a piece of software runs on increases the software's desirability to attackers. \\
			Number of Identities &	increases &	Asset Impact	 &  (Proposed) The market for personal identities and credit card information suggests that the number of identities a piece of software manages increases the software's desirability to attackers.\\
			Number of Dollars &	increases &	Asset Impact	 & (Proposed) The amount of financial resources a piece of software manages increases the software's desirability to attackers\\
			Source Code Availability	& influences &	Asset Impact & While Anderson ~\cite{anderson2002security} argues that attack and defense
			are helped equally by the open vs. closed source decision, we collect this data to enable further analysis.  \\
			Confidentiality, Integrity, Availability Requirements &	increases &	Asset Impact	& Explicit security requirements for a piece of software imply a higher level of Asset Impact for the software ~\cite{mell2007complete}. \\
			\hline \\[-1.8ex]
			Team Location &	influences &	Adherence	& (Proposed)  Kocaguneli ~\cite{kocaguneli2013distributed} reports on the debate over the effect of team location on software quality, collecting data on team location supports study of its effect. \\
			Methodology	& influences &	Adherence	& Different risks are associated with different software methodologies~\cite{williams2004xpef,jones2000software} \\
			Apply Data Classification Scheme & increases & 	Adherence & (Proposed) Identifying data in need of protection supports reducing Software Risk~\cite{morrison2017surveying}.\\	
			Apply Security Requirements	&	increases	&	Adherence & (Proposed)  supports reducing Software Risk[ref Riaz, etc] ~\cite{morrison2017surveying}.\\
			Perform Threat Modeling &	increases &	Adherence &(Proposed) Identification and analysis of threats supports reducing Software Risk~\cite{morrison2017surveying}. \\	
			Document Technical Stack &	increases &	Adherence & (Proposed) Understanding and controlling platform and dependency characteristics supports reducing Software Risk~\cite{morrison2017surveying}.\\	
			Apply Secure Coding Standards &	increases	& Adherence & (Proposed)  Avoiding known implementation erros supports reducing Software Risk~\cite{morrison2017surveying}.\\
			Apply Security Tooling &	increases &	Adherence & (Proposed)  Automated static and dynamic security analysis supports reducing Software Risk~\cite{morrison2017surveying}.\\
			Perform Security Testing &	increases &	Adherence & (Proposed)  Explicit validation of security requirement fulfillment supports reducing Software Risk~\cite{morrison2017surveying}.\\	
			Perform Penetration Testing &	increases &	Adherence	& (Proposed)  Exploratory testing of security properties supports reducing Software Risk~\cite{morrison2017surveying}.\\
			Perform Security Review &	increases &	Adherence	&  McIntosh et al. ~\cite{mcintosh2014the} observed lower defects for highly reviewed components. Meneely et al. ~\cite{meneely2014empirical} observed lower vulnerabilities for components with experienced reviwers. \\
			Publish Operations Guide &	increases	& Adherence & (Proposed) Documenting software security characteristics and configuration requirements supports reducing Software Risk~\cite{morrison2017surveying}.\\
			Track Vulnerabilities &	increases &	Adherence & (Proposed) Incident recognition and response supports reducing Software Risk~\cite{morrison2017surveying}.\\	
			Improve Development Process &	increases &	Adherence & (Proposed)  Adoption and adaptation of security tools and techniques based on experience supports reducing Software Risk~\cite{morrison2017surveying}.\\	
			Perform Security Training &	increases &	Adherence	& (Proposed) Development team knowledge of security risks and mitigations supports reducing Software Risk~\cite{morrison2017surveying}.\\		
			\hline \\[-1.8ex] 
			Vulnerabilities	& represent & Outcomes &  Vulnerabilities are, by definition, a negative security outcome, e.g. ~\cite{alhazmi2007measuring}.\\
			Defects & represent & Outcomes	& Zhang~\cite{zhang2014towards} identifies defect tracking as a key context factor.\\		
			\hline \\[-1.8ex] 
			\hline \\[-1.8ex] 
		\end{tabular} 
	\end{scriptsize}
\end{table*} 
%\subsubsection{Software Risk}
%\subsubsection{Asset Impact}
%\subsubsection{Practice Adherence}
%\subsubsection{Security Outcomes}
\section{Methodology}
\label{sec:methodology}

In this section, we present the steps required for analyzing a data set in terms of the security outcomes theoretical model.
\begin{itemize}
\item Subject Selection
Select a dataset containing measurements of variables that can be related to the structural model's Asset Value, Software Risk, adherence, and outcomes constructs.  

\item Link data source variables to model constructs 
Use theoretical relationships to link measurement model data source variables to structural model relationships. Consider each variable in the dataset, and assign eligible variables to the appropriate model construct. For example, if the dataset contains a code churn metric, associate it with the Software Risk construct.
 
\item Estimate SEM model 

Encode the combined structural and measurement models in a SEM modeling tool, and run the tool to obtain estimates for the model.
\item Report model fit measures recommended by Kline~\cite{kline2015principles}:
\begin{itemize}
	\item Model chi-square with degrees of freedom and p-value.
	\item Stieger-Lind Root Mean Square Error of Approximation (RMSEA) - RMSEA is a `badness-of-fit' measure, where values less than 0.10 indicate acceptable fit.
	\item Bentler Comparative Fit Index (CFI) - CFI compares fit of the researcher's model to a baseline model, where values of 0.90 or greater are viewed as acceptable fit.
	\item Standardized Root Mean Square Residual (SRMR) - SRMR is a `badness-of-fit' measure of the difference between the observed and predicted correlations. Zero is a perfect score, scores below 0.08 are viewed as sufficient.
	\item Akaike Information Criteria (AIC)
	\item 
\end{itemize}

\item Model Fit and Re-specification
% TODO [Paragraph explaining model fit and indicators]

If model fit indicators show poor fit between the data and the model, consider adding, dropping, or moving measurement variables, if theory supports doing so. In the present study, we do not alter the constructs or relationships in the structural model, but we do allow a list of transformations for the measurement model, as follows:
\begin{itemize}
	\item Choice of measurement variable to construct association is at the discretion of the researcher. 
	\item Where more than one measurement variable measures the same concept (e.g. team size measured by a count and by an ordinal variable), variable choice is at the discretion of the researcher. 
	\item SEM requires measurement model variable variances to be within a narrow range of each other. Transforming a variable by taking its log, square root, or multiplying by a constant is at the discretion of the researcher (transformation must be documented).
\end{itemize} 

\item Report Results
% See page 464 of Kline for details
Report model fit in both the terms of the global fit indicators, and in terms of comparison between the expected theoretical relationships embedded in the model and the actual parameter magnitudes and signs observed in the data.

\end{itemize}

\section{Evaluation}
\label{sec:evaluation}

This section describes how we evaluated our model through two case studies of existing software development security data.

We presented the data elements to be collected for our full model in Section ~\ref{sec:model_measurement}, and the data collection guide for the measurement model gives instructions on how to collect the data for a software development project~\cite{morrison2016spefsite}.  SEM is a large-sample technique, with median sample size in the literature of 200 cases~\cite{kline2015principles}. The need for large quantities of software development security data leads us to examine existing software development security datasets. In addition to the practical considerations of requiring large amounts of data, confirmation of our hypothesized structural and measurement relationships in data we did not author would strengthen the case for the theorized relationships.
We have identified two candidate data sets, of increasing detail and decreasing observation count: 
\begin{itemize}
\item The National Vulnerability Database contains vulnerability records for a wide variety of software, over a long timespan.
\item The Core Infrastructure Initiative project census contains high-level project data for over 400 Debian~\footnote{https://www.debian.org/} packages.
\end{itemize}

We give further details for each of these datasets in their case study sections, below. We used the R~\footnote{https://www.r-project.org}  lavaan~\cite{roseel2012lavaan} (LAtent VAriable ANalysis) package to conduct our SEM analysis, as well as the ggplot2, semPlot and psych R packages.

\subsection{National Vulnerability Database Case Study}

The U.S. National Institute of Standards and Technology (NIST) maintains the National Vulnerability Database (NVD) ~\footnote{https://nvd.nist.gov/}, an online database of publicly reported software vulnerabilities, with over 79,000 vulnerabilities dating back to 1988. Vulnerability reporters assign each vulnerability a Common Vulnerability Scoring System (CVSS) score and associated CVSS base metrics, according to the scheme defined in the CVSS guide~\cite{mell2007complete}.  

\subsubsection{Data selection}
 We translate the CVSS metrics into the terms of our model constructs and measurements. For each structural model construct, we present our metric associations and the rationale behind them.
 
\textbf{Asset Value}
The CVSS Confidentiality Impact, Integrity Impact, and Availability Impact metrics measure the degree of loss of confidentiality, integrity, and availability represented by a reported vulnerability~\cite{mell2007complete}. The CVSS Impact values:
	\begin{itemize}
		\item 'None' indicates no impact, 
		\item 'Partial' indicates partial impact, and 
		\item 'Complete' indicates 'Complete' impact to the CIA property of the system by the exploited vulnerability.  
	\end{itemize}
We model each Impact metric as a component of Asset Value, theorizing that confidentiality, integrity, and availability impacts change based on the usage context in which the software is run, e.g. the value of the assets managed. We translate None/Partial/Complete to an ordinal scale to model increasing CIA impact risk.

\textbf{Software Risk}

The CVSS Access Vector, Access Complexity, and Authentication metrics capture how a vulnerability is accessed and whether or not extra conditions are required to exploit it.~\cite{mell2007complete}. We translate these metrics into the terms of our model constructs and measurements. For each metric, we now quote the CVSS guide definition, and give a rationale for the associations we have defined:
\begin{itemize}
	\item Access Vector - measures how the vulnerability is exploited, in terms of network distance. Values:
	\begin{itemize}
		\item 'Local' requires the attacker to have physical access or an account on the system, \item 'Adjacent Network' requires access to the physical network on which the vulnerable software resides, and \item 'Network' requires only logical network acccess, e.g. remote access. 
	\end{itemize}
	We model Access Vector as a component of Software Risk, theorizing that the vector value changes based on design choices made by the software development team. We translate Local/Adjacent Network/Network to an ordinal scale, 1, 2, 3 to model increasing `Access Vector Risk'.
	\item Access Complexity - measures the complexity of attack required to exploit a vulnerability once an attacker has gained access. Values: 
	\begin{itemize}
		\item 'High' indicates specialized/elevated access is required to exploit the vulnerability, 
		\item 'Medium' indicates that some access is required to exploit the vulnerability, and \item 'Low' indicates that the vulnerability can be exploited with default/no special access.  
	\end{itemize}
	We model Access Complexity as a component of Software Risk, theorizing that the complexity value changes based on design choices made by the software development team. We translate High/Medium/Low to an ordinal scale, 1, 2, 3 to model increasing `Access Complexity Risk'.
	\item Authentication - measures the number of times an attacker must authenticate to exploit a vulnerability. Values: 
	\begin{itemize}
		\item 'Multiple' indicates an attacker must authenticate two or more times to exploit the vulnerability, 
		\item 'Single' indicates that an attacker must authenticate once to exploit the vulnerability, and 
		\item 'None' indicates that the vulnerability can be exploited without authenticating.  
	\end{itemize}
	We model Authentication as a component of Software Risk, theorizing that the authentication requirement changes based on design choices made by the software development team. We translate Multiple/Single/None to an ordinal scale, 1, 2, 3 to model increasing `Authentication Risk'.
\end{itemize}
  
\textbf{Adherence}
We model Adherence via the passage of time; we take the year each vulnerability is reported as indicative of the state of security practice adherence at the time the vulnerability was reported. We define an `adherence' metric for each vulnerability as the difference, in years, between the vulnerability's publication date and the initial year in the NVD database. If software quality (presumed to be caused by practice adherence) is increasing, we would expect to see a positive correlation between year and the CVSS impact metrics.  In an earlier test of this hypothesis, Kaminsky et al. ~\cite{kaminsky2011showing} fuzzed ten years of office software releases, confirming that software quality had improved over the previous decade, as measured by crash counts per software release over time. 
    
\textbf{Outcomes}
We obtain a metric for Outcomes by treating each unique software name in the NVD records as a distinct project, and summing all vulnerabilities for a project, reflecting our theorized post-release vulnerability count metric. We group each project by its (software) name, and by publication year, taking the mean of the remaining CVSS metrics to represent that project/publication year. 

% plot vuln count over time
% plot cvss score average
% plot three impact metrics
% plot three access metrics

	
\subsubsection{Data collection}
We collected the entire NVD dataset as of February 2017, but limit our analysis to complete years, from the year 2001 to 2016. The dataset includes vulnerabilities dating to 1988, but years previous to 2001 had relatively low activity (we measured activity for a year as the ratio of vulnerabilities reported that year to the average vulnerabilities per year over the NVD history. Activity first exceeded half of the overall average in 2001.) The years used, with vulnerability counts per year, are listed in Table ~\ref{nvd_vulns_year}.

\begin{table}
	\begin{center}	
		\caption{NVD Vulnerabilities, by year}
		\begin{large}	
			\begin{tabular}{l|l||l|l}
				$2001$ & 1672 & $2009$ & 5731\\
				$2002$ & 2155 & $2010$ & 4639\\
				$2003$ & 1524 & $2011$ & 4150\\
				$2004$ & 2445 & $2012$ & 5289\\
				$2005$ & 4927 & $2013$ & 4856\\
				$2006$ & 6608 & $2014$ & 7546\\
				$2007$ & 6516 & $2015$ & 6415\\
				$2008$ & 5632 & $2016$ & 6430\\
				%\bottomline
			\end{tabular}
			\label{tab:nvd_vulns_year}
		\end{large}
	\end{center}
\end{table}

% stargazer(rtrunc)
% Table created by stargazer v.5.2 by Marek Hlavac, Harvard University. E-mail: hlavac at fas.harvard.edu
% Date and time: Sun, Feb 26, 2017 - 22:57:53
\begin{table}[!htbp] \centering 
	\caption{NVD Project Demographics} 
	\label{tab:nvd_demog} 
	\begin{small}
	\begin{tabular}{@{\extracolsep{5pt}}lccccc} 
		\\[-1.8ex]\hline 
		\hline \\[-1.8ex] 
		Statistic & \multicolumn{1}{c}{N} & \multicolumn{1}{c}{Mean} & \multicolumn{1}{c}{St. Dev.} & \multicolumn{1}{c}{Min} & \multicolumn{1}{c}{Max} \\ 
		\hline \\[-1.8ex] 
		CVECount & 10,622 & 4.311 & 5.241 & 2 & 49 \\ 
		logCVECount & 10,622 & 1.466 & 0.533 & 1.099 & 3.912 \\ 
		cvss\_score & 10,621 & 6.139 & 1.429 & 1.200 & 10.000 \\ 
		adherence & 10,621 & 4.678 & 1.007 & 0.225 & 6.532 \\ 
		cvss\_auth & 10,622 & 2.905 & 0.224 & 0.000 & 3.000 \\ 
		cvss\_access\_vector & 10,622 & 2.794 & 0.486 & 0.000 & 3.000 \\ 
		cvss\_access\_complexity & 10,622 & 2.577 & 0.409 & 0.000 & 3.000 \\ 
		cvss\_conf\_impact & 10,622 & 1.833 & 0.504 & 0.000 & 3.000 \\ 
		cvss\_integ\_impact & 10,622 & 1.902 & 0.467 & 0.000 & 3.000 \\ 
		cvss\_avail\_impact & 10,622 & 1.833 & 0.552 & 0.000 & 3.000 \\ 
		\hline \\[-1.8ex] 
	\end{tabular} 
		\end{small}
	
\end{table} 

\subsubsection{Estimation}

Combining the structural and measurement models we have defined with the CVSS data collected from the NVD database, we have the model definition, expressed in lavaan syntax: 

\begin{align}
	$SoftwareRisk =\sim  cvss\_access\_vector +\\ cvss\_access\_complexity + $cvss\_auth$\\
	Outcomes $=\sim  logCVECount$\\
	$Outcomes \sim SoftwareRisk + Adherence + AssetValue$\\
	$Adherence =\sim adherence$\\
	$SoftwareRisk \sim Adherence$\\
	$AssetValue =\sim cvss\_conf\_impact + cvss\_integ\_impact + $cvss\_avail\_impact$\\	
\end{align}		

\subsubsection{Model Fit and Re-specification}
SEM depends on the variances of the measured variables to be within an order of magnitude of each other. The base adherence and CVE Count metrics had variances two magnitudes larger than the other variables. We scaled adherence, and took the log of CVE Count to bring them within range of the other variables. The variation among the 22,000+ projects with exactly one vulnerability caused numeric problems for the estimation algorithm, and we elected to treat the group as outliers and drop them from consideration. We, further, excluded projects with 50 or more vulnerabilities as outliers (79 projects). After the exclusions, 10621 projects remained in the analyzed dataset. 

\subsubsection{Reporting Results}

Table \ref{tab:results_fit_all} presents the fit measure results for the NVD case study (as well as for the other case studies).

\begin{table*}
	\begin{center}	
		\caption{Global Fit Measures and Results}
			\label{tab:results_fit_all}
			\begin{tabular}{p{3cm}p{1cm}|p{2cm}p{2cm}p{2cm}p{2cm}}
				\\[-1.8ex]\hline 
				\hline \\[-1.8ex] 
				Fit Measure &	Threshold & NVD	& CII	& Combined No Net  & Combined Net \\
				\hline \\[-1.8ex] 				
				Number of observations &  & $10621$ & $42$  & 49 & 19 \\				
				Model chi-square &  & $2759.88$ & $42$ & 413.67 & 159.18\\				
				Model d.f. &  & $17$ & $42$ & 105 & 105\\		
				Model p-value & $\leq 0.01$ & $0.0$ & $42$ & $0.0$ & $0.0$\\
				RMSEA & $\leq 0.10$ &  $0.12$  & $42$ & 0.09 & 0.212 \\
				CFI & $> 0.90$ & $0.83$ & $42$ & 0.889 & 0.616\\
				SRMR & $< 0.08$ & $0.08$ & $42$& 0.12  & 0.178 \\
				\hline \\[-1.8ex] 				
			\end{tabular}
	\end{center}
\end{table*}

We report the estimated parameter values for the NVD structural and measurement models in \ref{tab:results_nvd}.

\begin{table}
	\begin{center}	
		\caption{NVD Initial Model Results}
		\label{tab:results_nvd}
		\begin{tabular}{l|rrrr}
				\\[-1.8ex]\hline 
				\hline \\[-1.8ex] 
			\textit{Latent Variables}:  & & & & \\  
			$\sim$ Measured variables& Estimate & Std.Err & z$-$value & $P(>|z|)$ \\
				\hline \\[-1.8ex]
			$SoftwareRisk =\sim$  & & & & \\                                   
			cvss\_ccss\_vctr   & 1.000 & &  & \\                             
			cvss\_ccss\_cmpl &  $-$2.19 &   0.235 &  $-$89.320 &   0.000\\
			cvss\_auth     &   $-$1.403  &  0.148  & $-$9.49   & 0.000\\
			$AssetValue =\sim$     & & & & \\                                    
			cvss\_conf\_mpct   & 1.000     & & & \\                       
			cvss\_intg\_mpct   & 0.798   & 0.011 & 70.746 &   0.000 \\
			cvss\_aval\_mpct  &  0.872   & 0.013 & 67.489   & 0.000 \\
			$Outcomes =\sim$    & & & & \\                                     
			CVECount     &  1.000  & & & \\                          
			$Adherence =\sim$   & & & & \\                                      
			adherence    &     1.000        & & & \\                    
			Regressions:  & & & & \\  
			%& Estimate & Std.Err & z$-$value & $P(>|z|)$ \\
			$Outcomes \sim$         & & & & \\                                     
			SoftwareRisk   &  73.07 &   39.52 & $-$0.013 &   0.992 \\
			Adherence       &  $-$2.84  &  6.829  &  -0.468  &  0.640\\
			AssetValue     &   0.623  &  0.0124  &  5.023 &   0.00\\
			$SoftwareRisk \sim$        & & & & \\                                  
			Adherence     &    0.046 &   0.005  &  7.579 &   0.000\\
			Covariances:  & & & & \\  
			%& Estimate & Std.Err & z$-$value & $P(>|z|)$ \\
			$AssetValue \sim\sim$          & & & & \\                                 
			Adherence      &  $-$0.073  &  0.010 &  $-$7.456 &   0.000\\
		\end{tabular}
	\end{center}
\end{table}

In terms of global fit, the NVD model was just outside the range of traditional fit criteria, as measured by the fit indexes. In terms of the parameter estimates for our hypothesized  construct relationships, we have the following:
\begin{itemize}
	\item  Asset Value - Security Outcomes is positive (0.623), as hypothesized.
	\item Software Risk - Security Outcomes is positive (73.08), as hypothesized. 
	\item Practice Adherence is slightly postitively associated (0.045) with Software Risk 
\end{itemize}	
	  Only the Asset Risk-Outcome estimate was statistically significant. 

\subsection{CII Case Study}

The Core Infrastructure Initiative was formed by the Linux Foundation to support critical elements of the global information infrastructure by identifying and funding projects in need of assistance~\footnote{https://www.coreinfrastructure.org/faq}. The CII team has developed a census of open source projects in potential need of assistance~\footnote{https://www.coreinfrastructure.org/programs/census-project}, as well as a program for measuring and rewarding compliance with good security practice~\footnote{https://bestpractices.coreinfrastructure.org/}.

\subsubsection{Data selection}
The CII census team has published the rationale behind their census and metrics ~\cite{wheeler2015open}, and has published their data and code on Github~\footnote{https://github.com/linuxfoundation/cii-census}. The 429 census records each contain data for one project deemed both important, and potentially at risk for security concerns, by the CII team according to their criteria.

The project record contains descriptive data, e.g. project name and version, and security-relevant metrics, e.g. lines of code, contributor count, as well as an overall `risk index' based on the CII team's estimation of the importance of the metrics used to compute the index value.  Parallel to our translations of the CVSS metric score text to ordinal values, we translate the CII $fact\_activity$, $fact\_age$, $fact\_comments$, and $fact\_team\_size$ columns from text descriptions to ordinal values. We also parse the risk index components field to extract the ordinal ranks the CII team assigned each project for the risk index components: `Website points', `CVE Risk', `12 month contributors', `Popularity', `Language', and (network) `Exposure'. We now present our mapping of the CII metrics to our constructs.

\textbf{Asset Value}
The CII package\_popularity field contains Debian package popularity contest~\footnote{http://popcon.debian.org/} results, a measure of the frequency of use of each package on each system, aggregated to the relative usage of the package within the Debian ecosystem. We reason that paclage popularity is proportional to our theorized `Number of Machines'  metric, and, so, should be modeled as a component of Asset Value.

\textbf{Software Risk}
We model total\_contributor\_count, Contributor Risk, and total\_code\_lines as components of Software Risk, on the rationale that multiple studies (e.g. ~\cite{,camilo2015do,dashevskyi2016on}) have identified team size and code size as contributors to software vulnerability. The CII metrics direct\_network\_exposure and process\_network\_data, and potential\_privilege\_escalation are similar in concept to the CVSS Access metrics, and we, similarly, model them as components of Software Risk, theorizing that their values change based on design choices made by the software development team. We do not use the CII risk index measures, as they are composites, and we wish to examine the behavior of each of the index components.

\textbf{Adherence}
CII census data related to Adherence includes measurements of recent developer activity, and documentation effort. Developer activity is measured in the CII census data via a count of developers committing within the last year, twelve\_month\_contrib, and in terms of an ordinal categorical variable, fact\_activity, which we label TeamActivity. Documentation effort is measured in the CII census data via ordinal categorical variables in the risk index description for the amount and quality of comments, fact\_comments, which we label CodeComments, and by the presence of a website documenting the project, which we label WebsitePoints.

\textbf{Outcomes}
CII census data related to Outcomes includes a count of CVE vulnerabilities reported against the project since 2010. 

\textbf{Field choice notes}
The CII dataset contains a set of essentially co-linear fields, for example CVE Risk is translation of the CVE count since 2010 for the project into a categorical variable. We preferred counts to categorical values in initial field choices for the model.

\subsubsection{Data collection}

We retrieved the CII census data results file, current as of July 2016, from the project Github repository~\footnote{https://github.com/linuxfoundation/cii-census/blob/master/results.csv}. 

% stargazer(ciiraw)

% Table created by stargazer v.5.2 by Marek Hlavac, Harvard University. E-mail: hlavac at fas.harvard.edu
% Date and time: Sun, Feb 26, 2017 - 23:20:47
\begin{table*}[!htbp] \centering 
	\caption{} 
	\label{} 
	\begin{small}
	\begin{tabular}{@{\extracolsep{5pt}}lccccc} 
		\\[-1.8ex]\hline 
		\hline \\[-1.8ex] 
		Statistic & \multicolumn{1}{c}{N} & \multicolumn{1}{c}{Mean} & \multicolumn{1}{c}{St. Dev.} & \multicolumn{1}{c}{Min} & \multicolumn{1}{c}{Max} \\ 
		\hline \\[-1.8ex] 
		CVE\_since\_2010 & 428 & 7.262 & 26.142 & 0 & 422 \\ 
		twelve\_month\_contributor\_count & 346 & 98.965 & 491.693 & 0 & 3,768 \\ 
		total\_contributor\_count & 346 & 428.974 & 1,939.063 & 1 & 14,821 \\ 
		total\_code\_lines & 346 & 1,144,234.000 & 2,907,285.000 & 120 & 18,237,262 \\ 
		package\_popularity & 424 & 128,884.800 & 54,885.200 & 1 & 175,853 \\ 
		direct\_network\_exposure & 428 & 0.178 & 0.383 & 0 & 1 \\ 
		process\_network\_data & 428 & 0.152 & 0.359 & 0 & 1 \\ 
		potential\_privilege\_escalation & 428 & 0.058 & 0.235 & 0 & 1 \\ 
		risk\_index & 428 & 6.624 & 2.501 & 1 & 13 \\ 
		CVE & 428 & 7.262 & 26.142 & 0 & 422 \\ 
		RiskIndex & 428 & 6.624 & 2.501 & 1 & 13 \\ 
		TeamActivity & 428 & 1.509 & 0.988 & 0 & 3 \\ 
		CodeAge & 428 & 3.040 & 1.624 & 0 & 4 \\ 
		CodeComments & 428 & 2.231 & 1.773 & $-$1 & 5 \\ 
		TeamSize & 428 & 2.236 & 2.302 & $-$1 & 5 \\ 
		WebsiteRisk & 428 & 0.131 & 0.338 & 0 & 1 \\ 
		CVERisk & 428 & 1.044 & 1.266 & 0 & 3 \\ 
		ContributorRisk & 428 & 1.722 & 1.981 & 0 & 5 \\ 
		PopularityRisk & 428 & 1.808 & 0.494 & 0 & 2 \\ 
		LanguageRisk & 428 & 1.360 & 0.934 & 0 & 2 \\ 
		ExposureRisk & 428 & 0.558 & 0.777 & 0 & 2 \\ 
		DataRisk & 428 & 0.000 & 0.000 & 0 & 0 \\ 
		\hline \\[-1.8ex] 
	\end{tabular} 
	\end{small}
\end{table*}

\subsubsection{Estimation}

Combining the structural and measurement models we have defined with the metric data collected from the CII census, we have the model definition, expressed in lavaan syntax, in Figure \ref{tab:cii_model}.


\begin{align}
$SoftwareRisk =\sim  total\_contributor\_count + total\_code\_lines +$\\ $direct\_network\_exposure + process\_network\_data +$\\ $potential\_privilege\_escalation + CodeAge + TeamSize + LanguageRisk$ \\
$Outcomes =\sim  CVE_since_2010$ \\
$Outcomes \sim SoftwareRisk + Adherence + AssetValue$ \\
$Adherence =\sim  TeamActivity + CodeComments + WebsiteRisk +$ $twelve\_month\_contributor_count$  \\
$SoftwareRisk \sim  Adherence$ \\
$AssetValue =\sim package\_popularity$ \\
\end{align}


\subsubsection{Model Fit and Re-specification}
 
Table \ref{tab:results_fit_cii} presents the fit measure results for the final models of each case study. The baseline model, including all CII data fields, had poor fit indices (e.g. RMSE 0.237). We applied transformations and dropped fields from the model according to our expected theoretical relationships, but did not achieve significantly better model fit for the combinations we tested.

\subsubsection{Reporting Results}

Table \ref{tab:results_fit_all} presents the fit measure results for the CII case study (as well as for the other case studies). We report the estimated parameter values in \ref{tab:results_cii}.

In terms of global fit, the CII model was well outside the range of traditional fit criteria, as measured by the fit indexes. We did not conduct hypothesis tests for the model. 

\subsection{Combined CII and NVD data}
We observed that a subset of the CII census projects have vulnerabilities recorded in the NVD dataset, offering the opportunity to study the combination of measurement variables from the two datasets.  

\subsubsection{Data selection}
We began with the NVD and CII data, collected as previously described. 

\subsubsection{Data collection}
We intersected the two datasets based on the project name used in each dataset. Intersecting the 428 CII projects with the 10621 NVD projects yielded a set of 69 projects with all data for each project.

\subsubsection{Estimation}
We began with the aggregated fields from both datasets, and removed theoretically equivalent fields, preferring ordinal and ratio fields to categorical fields where we had a choice. For example, we chose the `package\_popularity' count over the `PopularityRisk' categorical variable. We evaluated model fit after each variable removal, addition, and transformation.

\subsubsection{Model Fit and Re-specification}
Through a series of experiments, evaluating model fit for the combined datasets, we obtained a `Combined' model with good global fit characteristics, see the `Combined 1' column in Table \ref{tab:results_fit_all}.

\begin{align}
	$SoftwareRisk =\sim cvss\_access\_vector + cvss\_access\_complexity$\\
	$ + cvss\_auth + total\_code\_lines + TeamSize + LanguageRisk$\\
	$Outcomes =\sim  logCVECount $\\
	$Outcomes ~ SoftwareRisk + Adherence + AssetValue$\\
	$Adherence =\sim adherence + twelve\_month\_contributor\_count +$\\ $CodeComments + TeamActivity $\\
	$SoftwareRisk \sim  Adherence$\\
	$AssetValue =\sim cvss\_conf\_impact + cvss\_integ\_impact +$\\ $cvss\_avail\_impact + package\_popularity$\\
\end{align}

\subsubsection{Reporting Results}
To study how models behave for different projects, we selected a non-model variable, `process\_network\_data', to filter data for the model. We compared the performance of the model on projects that do not process network data against the model's performance in projects that process network data by filtering the data each way and applying the model.

\begin{table}
	\begin{center}	
		\caption{NVD-CII Combined, No Net Data}
		\label{tab:results_combined1}
		\begin{tabular}{l|rrrr}
			\\[-1.8ex]\hline 
			\hline \\[-1.8ex] 
			\textit{Latent Variables}: &  & Combined & No Net & \\  
			$\sim$ Measured variables& Estimate & Std.Err & z$-$value & $P(>|z|)$ \\
			\hline \\[-1.8ex]
			$SoftwareRisk =\sim$  & & & & \\                                   
			cvss\_ccss\_vctr   & 1.000 & &  & \\                             
			cvss\_ccss\_cmpl &  $-$0.36 &   0.286 & 1.243 &   0.214\\
			cvss\_auth     &   $-$0.065  &  0.056  & $-$1.17   & 0.244\\
			total\_code\_lns  &  6.232 &   2.572 &   2.423 &   0.015\\
			TeamSize        &  6.268   & 2.484   & 2.524   & 0.012\\
			LanguageRisk    &  0.148  &  0.413   & 0.357   & 0.721\\ 
			& & & & \\  
			$AssetValue =\sim$     & & & & \\                                    		
		    cvss\_conf\_mpct  &  1.000  &	&	&                  \\
		    cvss\_intg\_mpct  &  0.954  &  0.116 &   8.230  &  0.000\\
		    cvss\_aval\_mpct  &  0.896 &   0.122 &   7.361 &   0.000\\
		    package\_pplrty  &  0.458 &   0.464  &  0.987 &   0.324\\	
			& & & & \\  
			$Outcomes =\sim$    & & & & \\                                     
			logCVECount     &  1.000  & & & \\                          
			& & & & \\  
			$Adherence =\sim$   & & & & \\                                      
			adherence    &     1.000        & & & \\    
			
		    twlv\_mnth\_cnt\_  &  7.783  &  4.088  &  1.904 &   0.057\\
		    CodeComments    &  1.914  &  1.145  &  1.672  &  0.095\\
		    TeamActivity    &  1.386  &  0.842 &   1.646 &   0.100\\	
			                
			Regressions:  & & & & \\  
			%& Estimate & Std.Err & z$-$value & $P(>|z|)$ \\
			$Outcomes \sim$         & & & & \\                                     
			SoftwareRisk   &  3.365 &   1.635 & 2.058 &   0.040 \\
			Adherence       &  $-$1.559  & 1.526  &  $-$1.022  &  0.307\\
			AssetValue     &   $-0$.619  &  0.291  &  $-$2.125 &   0.034\\
			$SoftwareRisk \sim$        & & & & \\                                  
			Adherence     &    1.56 &   1.04  &  1.505 &   0.132\\
			Covariances:  & & & & \\  
			%& Estimate & Std.Err & z$-$value & $P(>|z|)$ \\
			$AssetValue \sim\sim$          & & & & \\                                 
			Adherence     &    $-$0.024 &   0.019  &  $-$.255 &   0.209\\
		\end{tabular}
	\end{center}
\end{table}


\begin{table}
	\begin{center}	
		\caption{NVD-CII Combined, Net Data}
		\label{tab:results_combined2}
		\begin{tabular}{l|rrrr}
			\\[-1.8ex]\hline 
			\hline \\[-1.8ex] 
			\textit{Latent Variables}: &  & Combined & No Net & \\  
			$\sim$ Measured variables& Estimate & Std.Err & z$-$value & $P(>|z|)$ \\
			\hline \\[-1.8ex]
			$SoftwareRisk =\sim$  & & & & \\                                   
			cvss\_ccss\_vctr   & 1.000 & &  & \\                             
			cvss\_ccss\_cmpl &  $-$0.224 &   0.272 & 0.822 &   0.411\\
			cvss\_auth     &   $-$0.1035  &  0.114  & $-$906   & 0.365\\
			total\_code\_lns  &  7.27 &   3.054 &   2.381 &   0.017\\
			TeamSize        &  8.448   & 3.092  & 2.732   & 0.006\\
			LanguageRisk    &  $-$0.976  &  0.645   & $-$1.513   & 0.130\\ 
			& & & & \\  
			$AssetValue =\sim$     & & & & \\                                    		
			cvss\_conf\_mpct  &  1.000  &	&	&                  \\
			cvss\_intg\_mpct  &  0.743  &  0.129 &   5.770  &  0.000\\
			cvss\_aval\_mpct  &  0.481 &   0.110 &   4.364 &   0.000\\
			package\_pplrty  &  0.378 &   1.099  &  0.344 &   0.731\\	
			& & & & \\  
			$Outcomes =\sim$    & & & & \\                                     
			logCVECount     &  1.000  & & & \\                          
			& & & & \\  
			$Adherence =\sim$   & & & & \\                                      
			adherence    &     1.000        & & & \\    
			twlv\_mnth\_cnt\_  &  3.852  &  0.951  &  4.051 &   0.000\\
			CodeComments    &  $-$0.486  &  0.353  &  $-$1.376  &  0.1695\\
			TeamActivity    &  0.123  &  0.352 &   0.350 &   0.726\\	
			
			Regressions:  & & & & \\  
			%& Estimate & Std.Err & z$-$value & $P(>|z|)$ \\
			$Outcomes \sim$         & & & & \\                                     
			SoftwareRisk   &  $-$6.314 &   4.033 & $-$1.566 &   0.117 \\
			Adherence       &  4.210  & 2.340  &  1.80  &  0.072\\
			AssetValue     &   $-0$1.312  &  0.418  &  $-$3.142 &   0.002\\
			$SoftwareRisk \sim$        & & & & \\                                  
			Adherence     &    -0.003 &   0.035  &  -0.086 &   0.931\\
			Covariances:  & & & & \\  
			%& Estimate & Std.Err & z$-$value & $P(>|z|)$ \\
			$AssetValue \sim\sim$          & & & & \\                                 
			Adherence     &    $-$0.003 &   0.035  &  $-$.086 &   0.931\\
		\end{tabular}
	\end{center}
\end{table}
\section{Discussion}
\label{sec:discussion}

\begin{itemize}
	\item The NVD data offers per-manufacturer, per-project overview of reporter interest in software vulnerabilities, evidence that per-vulnerability severity is decreasing, and confirmation that the network is an important attack vector.
	\item The CII data offers evidence that SLOC and team size are correlated with manifest vulnerabilities. 
	\item Combined NVD-CII data supports better model fit, and offers theoretical support for X, Y, Z elements of the SP-EF measurement model. 
	\item Stepping back from the specifics of the case studies, SEM offers means of assessing the relative importance of the measurements taken for software security assessement.
	\item Asset Value, Software Risk, Adherence, and Outcomes appear to be a reasonable place to start for organizing studies of the factors behind software security success and failure. 
\end{itemize}
\section{Limitations}
\label{sec:limitations}
We now discuss threats to validity of our study

\subsection{External Validity}
Our two datasets represent thousands of open source and commercial software projects. However, the datasets each contain subsets of the variables we theorize are necessary to assess security posture.   

Kaminsky~\cite{kaminsky2011showing} critiqued the NVD data, pointing out that the existence of a vulnerability record is more indicative of reporter and finder interest in the software more than of the software's quality. We view reporter and finder interest as indicative of the kind of asset risk we seek to measure, distinct from software quality. Further work comparing software quality between samples of non-NVD projects and NVD projects is needed to establish the strength of the 'interest effect' and its effect on asset risk.

The variety of factors involved in security measurement suggest that further investigation is necessary. Complete validation of the model would require use of a variety of frameworks and data sources to evaluate the constructs and their relationships. That said, we used two independent data sources, increasing confidence in correlations found in both data sets. 

\subsection{Construct Validity}
In terms of construct validity, we propose a structural model of factors we believe to be relevant, and a measurement model based on the literature, but we leave room for augmenting the existing set of factors and the measurements taken on those factors. The analytical tools of SEM provide diagnostics to check for residual error and modification potential, enabling iteration over the structural and measurement models to account for unforeseen factors in the model.  

%[Add refs to Mell/Scarfone Improving CVSS and An Analysis of CVSS scoring, add discussion of limits of NVD use of web forms for reporting vulns/cvss.]
  
\subsection{Internal Validity}

The two datasets we used each contain subsets of the variables we theorize are necessary to assess security posture. We expect that the missing variables influence both the relative measures of each factor, and of the relationships between each factor. 

Statistical model-building in software engineering often uses Bayesian Belief Networks rather than SEM, e.g. [cite Fenton]. Judea Pearl has claimed the two techniques are essentially identical, preferring SEM when the research question if of the form `What factors determine the value of this variable' - \footnote{\url{http://causality.cs.ucla.edu/blog/index.php/2012/12/07/on-structural-equations-versus-causal-bayes-networks/}} We view our task in terms of determining the factors behind the values of the modeled variables, leading us to cast the model in terms of SEM.
\section{Conclusion}
\label{sec:conclusion}

In this paper, we have presented a model of factors affecting software security, with empirical tests of the model using two datasets. Our results indicate:
\begin{itemize}
	\item In the OpenHub-NVD data, Usage Risk, as measured by user count, has a comparable correlation with Outcomes to Development Risk, as measured by SLOC, Churn, Contributor Count, and project age. Factors outside the team's direct influence have to be considered when evaluating security performance and mitigations. 
	\item Our data corroborate previous research findings that team size and code churn are correlated with discovered vulnerabilities, and do so while controlling for other factors influencing security Outcomes. Measuring the relative impact of each measurement on its construct, and on the model's performance as a whole, supports refining the measurement framework and the theoretical model as further data are collected and evaluated.
	\item Stepping back from the specifics of the case studies, SEM offers means of assessing the relative importance of the measurements taken for software security assessment.
	\item Usage Risk, Development Risk, Adherence, and Outcomes appear to be a reasonable place to start for organizing studies of the factors behind software security success and failure. 
\end{itemize}

Our data suggest that not only software attributes, but the context of software use must be accounted for to assess the state of software security. Researchers and practitioners should measure both software attributes and software usage context when assessing software development practice adherence. That said, our analysis shows correlation, but not causation. Further work including manipulation of variables must be conducted to assess the causes of software insecurity and security. 

\section{Acknowledgements}

This work was supported through an IBM PhD Fellowship, and the USA National Security Agency (NSA) Science of Security Lablet. Any opinions expressed in this material are those of the author(s) and do not necessarily reflect the views of IBM or the NSA. We would like to thank the North Carolina State University Realsearch group for their helpful comments on the paper. We would like to thank Dr. Chris Parnin for his helpful feedback.

%\input{samplebody-conf}

\bibliographystyle{ACM-Reference-Format}
\bibliography{modeling} 

\end{document}
