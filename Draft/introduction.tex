\section{Introduction}
\label{sec:intro}

In February 2017, security researchers discovered a programming bug in an HTML parser at Cloudflare\footnote{https://blog.cloudflare.com/incident-report-on-memory-leak-caused-by-cloudflare-parser-bug/} . The parser was sending decrypted encryption keys and passwords along with ordinary web requests, resulting in unencrypted sensitive data being cached across the Internet. While the programming bug, now known as `Cloudbleed'~\footnote{https://blog.cloudflare.com/quantifying-the-impact-of-cloudbleed}, was on a single line of code, the bug had widespread effects because of the code's context, running on many servers, and sending sensitive results to many clients. While researchers have surfaced relationships between source code metrics and vulnerabilities~\cite{zimmerman2010searching,alhazmi2007measuring,meneely2013when,shin2011evaluating}, the example of Cloudbleed suggests that software development teams must consider software's environment and data as well as the software itself when evaluating the potential for vulnerabilities. Development teams and researchers could benefit from a comprehensive picture of the factors underlying security concerns.
\textit{The goal of this research is to support researchers and practitioners in measuring the effect of software quality context factors, software usage context factors, and security practice adherence on software security outcomes by building and evaluating an explanatory model using structural equation modeling.}

We propose and evaluate a model for quantifying security practice use and outcomes during software development. The four constructs of our model are:
\begin{enumerate}
	\item Software Risk - measures of the characteristics of software that make successful attacks more likely;	
	\item Asset Impact - measures of the value an attacker will find in conducting a successful attack;
	\item Practice Adherence (Adherence) - measures of the development team's security assurance efforts; 
	\item Security Outcomes (Outcomes) - measures of security-related events associated with a piece of software over the course of the software's life cycle.
\end{enumerate}

We hypothesize that the four constructs are related as follows:
\begin{itemize}
	\item \textbf{H1} Asset Impact is positively associated with negative Security Outcomes
	\item \textbf{H2} Software Risk is positively associated with negative Security Outcomes
	\item \textbf{H3} Practice Adherence is negatively associated with Software Risk 	
\end{itemize}
	
To evaluate the model and our hypotheses, we conduct two quantitative studies of the construct relationships, applying data from the National Vulnerability Database (NVD) and the Core Infrastructure Initiative (CII) Census to test our hypotheses.
  
Our contributions include:
\begin{itemize}
\item An explanatory model for how software development security practice use and security risk affect security outcomes;
\item Two case study evaluations on how our model fits collected data from software projects, and
\item a set of guidelines for applying our model to new software projects. 
\end{itemize}
The remainder of this paper is organized as follows:  Section \ref{sec:background} discusses background and related work. Section \ref{sec:model} presents an overview of our model and its underlying hypotheses. Section \ref{sec:methodology} presents our study methodology. Section \ref{sec:evaluation} presents the case studies and their results. Section \ref{sec:discussion} discuses the measurement results. Section \ref{sec:limitations} presents our study limitations. Section \ref{sec:conclusion} presents our conclusion.