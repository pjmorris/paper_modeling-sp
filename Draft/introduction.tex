\section{Introduction}
\label{sec:intro}
Software development teams attend to security during the development process to minimize vulnerabilities in the software they produce. Many practices for controlling and measuring software security quality have been proposed and evaluated by the research community, for example design patterns~\cite{uzunov2015comprehensive}, static and dynamic analysis~\cite{austin2013comparison}, and attack surface analysis~\cite{theisen2015approximating}. Practitioner recognition of security practices can be seen in the appearance of published lists of security practices, for example the Building Security in Maturity Model~\cite{mcgraw2013bsimm} (BSIMM). However, the BSIMM does not provide practice recommendations. Software development teams must collect and evaluate their own evidence for what practices are suitable for a given software development project. 
At the same time, research into vulnerability prediction has yielded metrics that can help teams identify whether the software they are developing may be at risk of attack. For example, the Common Infrastructure Initiative (CII) census of open source projects needing security investments, Wheeler ~\cite{wheeler2015open}. The census researchers identified common metrics linked to vulnerabilities, for example Source Lines of Code (SLOC), and produced a list of at-risk software, but did not investigate the practices employed by the software projects in the census. At present, it is difficult for a software development team to assess whether their security practices are proportional to their vulnerability risk.  
\textit{The goal of this research is to support researchers and practitioners in measuring the effect of security practice use on software vulnerabilities by building and validating an explanatory model for software development security practice use.} We propose and evaluate a model for quantifying security practice use and outcomes during software development. The three central constructs of our model are:
\begin{itemize}
\item Security criticality - Security criticality represents how much value an attacker will find in conducting a successful attack.
\item Security effort - Security effort represents the development team’s security assurance efforts, the summation of all of the specific techniques the team applies for the sake of security. 
\item Security outcomes - Security outcomes represent the security vulnerabilities discovered over the course of the software’s lifecycle.
\end{itemize}
To test our theoretical model, we investigate three hypotheses, framed as research questions:
\begin{itemize}
\item RQ1: Is security effort proportional to favorable security outcomes?
\item RQ2: Is security criticality proportional to adverse security outcomes?
\item RQ3: Is security criticality proportional to security effort?
\end{itemize}

To specify our model, we follow the guidelines of Structural Equation Modeling (SEM) [13]. SEM is a family of statistical techniques for specifying and assessing models for causal inference. Causal inference [14], [15] augments standard statistical methods with tools for indicating assumptions about cause and effect, and testing the conditions under which those assumptions hold. Two features of SEM make it suitable to our purposes; latent variables and causal inference. Latent variables [13] represent hypothetical (‘plausible’) constructs, e.g. ‘intelligence’ or ‘security criticality’, that cannot be directly measured, but that can be indirectly measured in terms of one or more observed variables. Causal inference enables reasoning about causality based on observational data.  
We structure data collection for our model using the Security Practices Evaluation Framework ~\cite{morrison2016spefsite} (SP-EF) . SP-EF is designed to enable assessing how software development security practices affect security outcomes, and to enable the development of a body of knowledge on the effect of security practice use in software development.
In this paper, we present the hypotheses underlying the SP-EF data elements and practices, and how they combine to form an explanatory model for security practice use and outcomes in software development. We then collect SP-EF data from two open source projects, OpenSSL and phpMyAdmin, to assess our data collection and analysis procedures for the model. 
Our contributions include:
\begin{itemize}
\item An explanatory model, expressed as a network of related variables and hypotheses, for how software development security practice use and security criticality affect security outcomes.
\item An empirical evaluation of how well our model fits collected data from software projects. 
\end{itemize}
The remainder of this paper is organized as follows:  \ref{sec:background} discusses background and related work. \ref{sec:model} presents an overview of our model and is underlying hypotheses. \ref{sec:evaluation} presents our study methodology. \ref{sec:results} presents the case study results. \ref{sec:discussion} discuses the measurement results. \ref{sec:limitations} presents our study limitations. \ref{sec:conclusion} presents our conclusion.

