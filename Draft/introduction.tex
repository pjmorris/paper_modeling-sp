\section{Introduction}
\label{sec:intro}

As evidenced by, for example, the Heartbleed~\footnote{\url{http://heartbleed.com/}} and Shellshock~\footnote{\url{http://tinyurl.com/mfv93td}} vulnerabilities, a single commit, sometimes even a single line of code, can cause a large-scale security problem.  Consequently, software development teams are investigating the use of security practices in software development. When presented with lists of the hundreds of software development security practices available (e.g. those listed in the `Building Security in Maturity Model ~\cite{mcgraw2013bsimm} (BSIMM)), the first question industry practitioners typically ask is `Where do I start?'  To answer this question requires an understanding of the questioner's most significant security problems, and a theory for how software development security practices affect the software. To support collection of the data needed to answer  questions about security practices and their effects empirically, we present a model for how security practice use affects security outcomes in software development.  Ultimately, our goal is to provide evidence-based answers for questions like `Where do I(we) start?', `What do we do next?', and `When do we stop?'.

 \textit{The goal of this research is to support researchers and practitioners in measuring the effect of security practice use on software vulnerabilities by building and validating an explanatory model for software development security practice use.}
  To measure security practice adherence and security criticality, we propose and evaluate a model for quantifying security practice use and outcomes during software development. The three central constructs of our model are:
\begin{itemize}
\item Criticality - Security criticality represents how much value an attacker will find in conducting a successful attack.
\item Adherence - Adherence represents the development team’s security assurance efforts, the summation of all of the specific techniques the team applies for the sake of security. 
\item Outcomes - Outcomes represent the security vulnerabilities discovered over the course of the software’s lifecycle.
\end{itemize}
To test our theoretical model, we investigate three hypotheses, framed as research questions:
\begin{itemize}
\item RQ1: Is adherence proportional to favorable security outcomes?
\item RQ2: Is criticality proportional to adverse security outcomes?
\item RQ3: Is adherence proportional to criticality?
\end{itemize}

To evaluate the model, we conduct a quantitative study of the construct relationships, and two qualitative case studies of the open source software projects OpenSSL, and phpMyAdmin.
  
Our contributions include:
\begin{itemize}
\item An explanatory model, expressed as a network of related variables and hypotheses, for how software development security practice use and security criticality affect security outcomes.
\item An empirical evaluation of how well our model fits collected data from software projects. 
\end{itemize}
The remainder of this paper is organized as follows:  Section\ref{sec:background} discusses background and related work. Section\ref{sec:model} presents an overview of our model and its underlying hypotheses. Section\ref{sec:evaluation} presents our study methodology. Section\ref{sec:results} presents the case study results. Section\ref{sec:discussion} discuses the measurement results. \ref{sec:limitations} presents our study limitations. \ref{sec:conclusion} presents our conclusion.
