\section{Methodology}
\label{sec:methodology}

In this section, we present the steps required for analyzing a data set in terms of the security outcomes theoretical model.
\begin{itemize}
\item Subject Selection
Select a dataset containing measurements of variables that can be related to the structural model's Asset Value, Software Risk, adherence, and outcomes constructs.  

\item Link data source variables to model constructs 
Use theoretical relationships to link measurement model data source variables to structural model relationships. Consider each variable in the dataset, and assign eligible variables to the appropriate model construct. For example, if the dataset contains a code churn metric, associate it with the Software Risk construct.
 
\item Estimate SEM model 

Encode the combined structural and measurement models in a SEM modeling tool, and run the tool to obtain estimates for the model.
\item Report model fit measures recommended by Kline~\cite{kline2015principles}:
\begin{itemize}
	\item Model chi-square with degrees of freedom and p-value.
	\item Stieger-Lind Root Mean Square Error of Approximation (RMSEA) - RMSEA is a `badness-of-fit' measure, where values less than 0.10 indicate acceptable fit.
	\item Bentler Comparative Fit Index (CFI) - CFI compares fit of the researcher's model to a baseline model, where values of 0.90 or greater are viewed as acceptable fit.
	\item Standardized Root Mean Square Residual (SRMR) - SRMR is a `badness-of-fit' measure of the difference between the observed and predicted correlations. Zero is a perfect score, scores below 0.08 are viewed as sufficient.
	\item Akaike Information Criteria (AIC)
	\item 
\end{itemize}

\item Model Fit and Re-specification
% TODO [Paragraph explaining model fit and indicators]

If model fit indicators show poor fit between the data and the model, consider adding, dropping, or moving measurement variables, if theory supports doing so. In the present study, we do not alter the constructs or relationships in the structural model, but we do allow a list of transformations for the measurement model, as follows:
\begin{itemize}
	\item Choice of measurement variable to construct association is at the discretion of the researcher. 
	\item Where more than one measurement variable measures the same concept (e.g. team size measured by a count and by an ordinal variable), variable choice is at the discretion of the researcher. 
	\item SEM requires measurement model variable variances to be within a narrow range of each other. Transforming a variable by taking its log, square root, or multiplying by a constant is at the discretion of the researcher (transformation must be documented).
\end{itemize} 

\item Report Results
% See page 464 of Kline for details
Report model fit in both the terms of the global fit indicators, and in terms of comparison between the expected theoretical relationships embedded in the model and the actual parameter magnitudes and signs observed in the data.

\end{itemize}
